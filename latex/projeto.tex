\documentclass[12pt, a4paper, tocpage=plain]{abnt} % Fonte tamanho 12, papel A4, páginas do sumário sem o p.<número da página>

\usepackage[brazilian]{babel} % Gera datas e nomes em português com estilo brasileiro
\usepackage{hyperref} % cria hyperlink deixando da mesma cor do texto (sem quadrado vermelho)
\usepackage[utf8]{inputenc} % Dá suporte para caracteres especiais como acentos e cedilha
\usepackage[T1]{fontenc}
\usepackage[alf]{abntcite} % Define o estilo de referência bibliográfica
\usepackage{graphicx} % Permite a utilização de imagens no documento
\usepackage[table,xcdraw]{xcolor} % permite colorir tabela
\usepackage{longtable}
\usepackage[small]{caption} % Define as legendas das figuras com fontes menores do que o texto
\usepackage{pslatex} % Define que o formato da letra será Times New Roman
\usepackage{epigraph} % Permite a criação de epígrafes
\usepackage{setspace} % Permite a definição de espaçamento entre linhas
\usepackage{nomencl} % define nomeclaturas para listas de siglas
\usepackage[top=3cm, left=3cm, right=2cm, bottom=2cm]{geometry} % Define as margens da folha

\setcounter{secnumdepth}{3} % Até três subsubsections numeradas
\setcounter{tocdepth}{3} % Até trẽs subsubsections numeradas
\setlength{\parindent}{1.25cm} % Define o recuo da primeira linha dos parágrafos para 1.25 cm

\usepackage{listings}
\usepackage{color}

\hypersetup{
  colorlinks=true,          % false: boxed links; true: colored links
  linkcolor=black,           % color of internal links
  citecolor=black,           % color of links to bibliography
  urlcolor=blue
}

\renewcommand{\ABNTchapterfont}{\bfseries} % Define a fonte do \chapter
\renewcommand{\ABNTchaptersize}{\large} % Define o tamanho da fonte do \chapter
\renewcommand{\ABNTsectionfontsize}{\large} % Define o tamanho da fonte da \section
\renewcommand{\ABNTsubsectionfontsize}{\large} % Define o tamanho da fonte do \subsection
\renewcommand{\ABNTsubsubsectionfontsize}{\large} % Define o tamanho da fonte do \subsubsection
\citeoption{abnt-repeated-author-omit=yes}
\bibliographystyle{abnt-alf}
\renewcommand{\ABNTbibliographyname}{REFERÊNCIAS BIBLIOGRÁFICAS} % Modifica o título gerado pelo \bibliographys


\begin{document} % Começo do TCC
\begin{titlepage}
 \begin{center}
   {\large UMA PERSPECTIVA REGIONAL DO GERENCIAMENTO DE PROJETOS EM UM POLO DE INOVAÇÃO} \\ [7cm]
   {\large PRISCILA MANHÃES DA SILVA} \\ [4cm]
   \vfill
   {\large UNIVERSIDADE ESTADUAL DO NORTE FLUMINENSE – UENF} \\ [1cm]
   {\large CAMPOS DOS GOYTACAZES – RJ} \\
   {\large JULHO – 2016}
 \end{center}
\end{titlepage}
 % Cria a capa
\newpage\null\thispagestyle{empty}\newpage % pagina em branco
\begin{titlepage}
 \begin{center}
   {\large UMA PERSPECTIVA REGIONAL DO GERENCIAMENTO DE PROJETOS EM UM POLO DE INOVAÇÃO} \\ [5cm]
   {\large PRISCILA MANHÃES DA SILVA} \\ [1.5cm]
   \hspace{.45\textwidth} % posicionando a minipage
   \begin{minipage}{0.5\textwidth}
   \begin{espacosimples}
        Projeto para dissertação submetido à Universidade Estadual do Norte Fluminense Darcy Ribeiro, como parte das exigências para obtenção do título de Mestre em Engenharia de Produção. \\[5cm]
    \end{espacosimples}
    \end{minipage}
    {\normalsize Orientador: Professor Doutor Rogério Atem Carvalho}\\
    {\normalsize Co-orientador: Professora Doutora Simone Vasconcelos Silva}
   \vfill
   {\large CAMPOS DOS GOYTACAZES – RJ} \\
   {\large JULHO – 2016}
 \end{center}
\end{titlepage}
 % Cria a folha de rosto
\begin{folhadeaprovacao}
  \setlength{\ABNTsignthickness}{0.4pt}
  \setlength{\ABNTsignwidth}{15cm}
  \setlength{\ABNTsignskip}{0.9cm}
  \begin{center}
    {\large UMA AVALIAÇÃO DO IMPACTO INICIAL DO USO DE FERRAMENTAS DE GESTÃO DE PROJETOS EM UM POLO DE INOVAÇÃO} \\ [3.5cm]
    {\large PRISCILA MANHÃES DA SILVA} \\ [1.5cm]
    \hspace{.45\textwidth} % posicionando a minipage
    \begin{minipage}{0.5\textwidth}
      \begin{espacosimples}
        Projeto para dissertação submetido à Universidade Estadual do Norte Fluminense Darcy Ribeiro, como parte das exigências para obtenção do título de Mestre em Engenharia de Produção.
      \end{espacosimples}
    \end{minipage}
  \end{center}
  {\normalsize Aprovada em 16 de agosto de 2016} \\\\
  {\normalsize Comissão Examinadora: }
  \bigbreak
  \assinatura{Profa. Annabell Del Real Tamariz (Doutor(a), Engenharia Elétrica) - UENF / LEPROD}
  \bigbreak
  \assinatura{Profa. Simone Vasconcelos Silva (Doutora, Computação) - IFF /  UENF / LEPROD \\ Coorientadora}
  \bigbreak
  \assinatura{Prof. Rogério Atem Carvalho (Doutor, Ciências de Engenharia) - IFF / UENF / LEPROD \\ Orientador}
\end{folhadeaprovacao}
 % Cria a folha de aprovação
% \include{dedicatoria}  % Cria a folha de dedicatória
% \begin{center}
\textbf{AGRADECIMENTOS}
\end{center}

Primeiramente agradeço ...
 % Cria a folha de agradecimentos
% \include{epigrafe} % Cria a epígrafe (onde se coloca um pensamento)
% \begin{center}
  \textbf{RESUMO}
\end{center}
\singlespacing

\noindent . \\

\noindent Palavras-chave:
 % Resumo do trabalho
% \begin{center}
  \textbf{ABSTRACT}
\end{center}

\singlespacing

\noindent . \\

\noindent Keywords:

 % Resumo em língua estrangeira

\singlespacing
\renewcommand{\listfigurename}{LISTA DE FIGURAS} % Modifica o nome da lista de figuras
\listoffigures % Gera o índice de figuras
\renewcommand{\listtablename}{LISTA DE TABELAS} % Modifica o nome da lista de tabelas
\listoftables % Gera o índice de tabelas
\makenomenclature
\renewcommand{\nomname}{LISTA DE SIGLAS} % Modifica o nome da lista de siglas
\printnomenclature % Gera o índice de siglas
% makeindex tcc.nlo -s nomencl.ist -o tcc.nls
\renewcommand{\contentsname}{SUMÁRIO} % Modifica o nome do sumário
\tableofcontents % Gera o sumário

\onehalfspacing % Define o espaçamento de 1.5cm entre linhas
\chapter{INTRODUÇÃO}
\thispagestyle{empty}

\section{Contextualização}
Com a chegada do século 21, o mundo enfrentou diversas mudanças, tanto com respeito as inovações tecnológicas quanto ao cenário no mercado produtivo. Custo e qualidade deixaram de ser a principal motivação de melhoria nas empresas, cedendo espaço para projetos que entreguem produtos mais flexíveis em menor espaço de tempo (Yusuf 2004).

Neste contexto foi possível destacar a importância da inovação, do atendimento as constantes modificações de requisitos por parte dos clientes e ainda importância de manter o cliente como alvo da organização, para conquistá-lo e gerar vínculos de fidelização (CHIESA, 2007).

O aumento da concorrência dos produtos nacionais com os equivalentes importados, também obriga as empresas a buscarem inovação na distinção de seus produtos frente a competitividade de seus concorrentes (ATKINSON et al. 2012).

Esse aumento do uso da inovação como principal fator de crescimento econômico traz a necessidade de uma forma de controle as regras de crescimento e a coordenação dos sistemas de inovação (SI) que tendem a surgir para que promovam a sustentabilidade e competitividade a longo prazo (LUNDVALL, 1998).

No Brasil, com a abertura do país ao comércio internacional, ocorre também o aumento dos incentivos governamentais para a pesquisa e desenvolvimento e a procura por inovação através de parcerias com universidades por parte das empresas resultando numa relação de aproximação. Cada vez mais as empresas, universidades e governos intensificam as suas interações através dos SI.

A empresa busca a inovação que permitirá a sua sobrevivência no mercado e o aumento de suas margens de lucros; a universidade encontra neste ambiente condições para a obtenção de mais financiamento para a pesquisa e também recursos para o financiamento, sua participação no sistema também acarreta uma maior aproximação com a sociedade na medida que os resultados da pesquisa tenham um aproveitamento mais direto pela mesma; enquanto isso, universidade e governos interagem em busca de resultados alinhados com os objetivos de cada um e, por derradeiro, os governos, que nas suas diversas esferas buscam o desenvolvimento local, regional e nacional e da sociedade como um todo.

Desde o final da década de 1990, as politicas de Ciência, Tecnologia e Inovação (CT\&I) vinham sendo implementadas no país seguindo a lógica de modelo linear de inovação. Juntamente, as politicas de Pesquisa e Desenvolvimento (P\&D) que visavam articular a relação entre universidades, empresas e governo através dos recursos previstos no Sistema Nacional de Inovação (SNI) (Associação Brasileira das Instituições de Pesquisa Tecnológica e Inovação [ABIPTI], 2001; KUHLMANN, 2008; GODIN, 2009).

Com o tempo, essas interações passaram a ser reconhecidas por interações de Pesquisa, Desenvolvimento e Inovação (PD\&I) e se responsabilizaram por gerar a produção de novos conhecimentos, proporcionando a transferência de tecnologia do meio acadêmico para as empresas e também a proteção destes ativos intelectuais através dos direitos de propriedade intelectual, mediante a obtenção de patentes e registro de direitos autorais. Ao incorporar o termo inovação, é possível destacar três aspectos fundamentais: interação com a sociedade; com as empresas; e com o governo. Em outras palavras, inovação significa P\&D mais transferência de tecnologia.

Os riscos mais comuns ao sucesso da relação UEG estão ligados as universidades que precisam se adequar para corresponder a demanda da sociedade; criar mecanismos de pesquisa e desenvolvimento interdisciplinares com múltiplas fontes de fomento, sejam governo, empresas ou instituições; estimular pesquisas prioritárias que aloquem necessidades de forma planejada; criar mecanismos de proteção a propriedade intelectual; lidar com a transferência das tecnologias criadas; e ainda estimular acadêmicos pelo caminho empreendedor a fim de gerar oportunidades.

Ao passo que a relação UEG passou a representar uma solução para a necessidade de inovação de produto no mercado, também houve uma crescente busca por soluções que tornassem possível gerenciar esta inovação em ambientes desafiadores e imprevisíveis, advindo dos processos da mesma, a indústria decidiu voltar-se ao gerenciamento de projetos (GP), que no entanto, de acordo com Geraldi (2008), mostrou-se incapaz de lidar com tais mudanças a princípio.

Para Lin (2006) as ferramentas e técnicas tradicionais empregadas no GP não previam atividades de inovação nem instabilidades durante o processo de seus produtos, pois previam projetos linear e sem inconstâncias.

Visando atender a expectativa de mercado e a melhoraria da gestão de projetos, abordagens alternativas, que vinham se desenvolvendo desde o início dos anos 90, com novos princípios, ferramentas e técnicas, pareceram a melhor resposta. Essas abordagens vieram a ser conhecidas por Gerenciamento Ágil de Projetos (GAP) (Amaral 2011).

Entretanto é preciso que haja melhor compreensão sobre cada abordagem no que respeito aos processos executados na formação dos produtos em seu meio, e verificada adaptação destas abordagens em meio a inovação.

\section{Problemática}

No mercado moderno a competitividade através da inovação está diretamente baseada no conhecimento e portanto na ciência e a tecnologia que são fatores básicos para a geração desta. A relação UEG, conhecida por tríplice hélice, se mostrou essencial para a transmissão do conhecimento de volta para o mercado e portanto para a sociedade.

Assim, cada vez mais as parcerias entre instituições de ensino direcionadas à pesquisa científica e tecnológica e empresas privadas têm ganhado destaque, visto a possibilidade de investimento de recursos para o desenvolvimento de soluções inovadoras para a sociedade.

Para garantir a viabilidade destas parceria e expor as vantagens obtidas nesta relação, diversas ferramentas devem ser postas em ação, entre essas ferramentas encontra-se o uso de abordagens de gerenciamento de projetos.

O uso de abordagens GAP aumentou consideravelmente desde sua oficialização através do Manifesto Ágil. A positividade demonstrada por estudos empíricos demonstrou que é possível utilizar estas abordagens para o sucesso de práticas de GP.

Alguns estudos têm apontado a possibilidade de melhorias através do seu uso em projetos de diversas naturezas, outros porém, destacam que continuam utilizando abordagens tradicionais em função de características específicas de projetos de grandes organizações. Assim não é possível determinar se o uso das abordagens GAP vai se espalhar por todos ramos controlando projetos ao longo do globo.

Existem também estudos que estão criando modelos de abordagens que utilizam tanto a base tradicional, quanto técnicas e ferramentas de abordagens GAP. Estes estudos estão utilizando sua própria base de conhecimento para modular seus modelos, entretanto existe vaga menção a aplicação desses modelos na prática através da realização de estudos de caso.

\section{Definição dos objetivos}

\subsection{Objetivo Geral}

Verificar através de um mapeamento e de um estudo de casos se os conceitos e ferramentas empregados no gerenciamento de projetos dentro da relação UEG podem realmente contribuir na busca por inovação e competitividade do mercado de trabalho.


\subsection{Objetivos específicos}

\begin{itemize}
  \item{Avaliar a filosófica das abordagens de gerenciamento ágil de projetos (GAP);}
  \item{Avaliar a diferença de produtividade implícita na filosofia de GAP;}
  \item{Levantar literatura científica sobre o tema;}
  \item{Observar em campo a aplicação da relação UEG, analisar e discutir seus resultados;}
  \item{Verificar a adequação no uso de práticas de GP;}
  \item{Verificar a satisfação das empresas envolvidas na relação UEG;}
  \item{Verificar a adequação do Polo de Inovação (PICG) frente ao plano de ação do governo brasileiro;}
  \item{Analisar o impacto regional frente a criação do PICG.}
\end{itemize}

\section{Justificativa}

Este trabalho apresenta uma visão sobre abordagens de gerenciamento de projetos, apontando vantagens e desvantagens de seu uso, e ainda sobre trabalhos que utilizaram técnicas de ambas abordagens, propondo novos modelos de seu uso. É possível argumentar que muitos projetos ainda não utilizam nenhuma abordagem, e que para os que utilizam abordagens tradicionais, a mesclagem com abordagens GAP são aconselháveis.

Recentemente a região de Campos do Goytacazes recebeu a abertura de um Polo de Inovação para facilitar na relação UEG. Observa-se um ambiente propício para esse tipo de interação a partir de algumas iniciativas que vem sendo promovidas pelo PICG, sendo notado também a necessidade de estudos aprofundados através de trabalhos futuros para análise do impacto da criação do polo.

A contribuição almejada neste trabalho será a definição de uma abordagem de análise para o processo de desenvolvimento do escritório de projetos do PICG, frente a uma abordagem mesclada de práticas de gerenciamento de projeto tradicionais e ageis, que visam possibilitar a entrega da inovação dentro de uma relação UEG.


\section{Estrutura do trabalho}

Este projeto está organizado em 5 capítulos. Neste primeiro capítulo foi apresentado o contexto no qual a pesquisa se encaixa frente as motivações do trabalho e seus objetivos.

No segundo capítulo são apresentadas as abordagens de gerenciamento de projetos existentes, relacionando sua criação e conceitos através de uma revisão bibliográfica.

O terceiro capítulo dá continuidade a revisão bibliográfica do projeto, porém relatando sobre a relação UEG, na sua criação e presença no país.

No quarto capítulo é apresentado o objeto do estudo de caso, realizado no Polo de Inovação de Campos dos Goytacazes (PICG).

O ultimo capítulo representa um espaço para discussão dos resultados a serem obtidos nesta pesquisa e portanto as considerações finais.


\chapter{REVISÃO DA LITERATURA}
\thispagestyle{empty}

\section{Conceitos Básicos}
\subsection{Projetos}

  \citeonline[p. 8]{meredith2011project} definem que projetos tratam da realização de tarefas específicas e finitas, de grande ou pequena escala, com um prazo de execução e um orçamento estipulado. \citeonline{turner2014handbook}, por sua vez, dispõe que projetos são tarefas com uma data final, onde caso esta data não implique na entrega do projeto, será estabelecida uma entrega do produto presente e a criação de outro projeto para entregar as tarefas restantes, referente a este produto.

  Para \citeonline{kerzner2013project} um projeto pode ser caracterizado por uma série de atividades e tarefas realizadas com um objetivo específico para serem completadas sob determinadas especificações. Essas atividades e tarefas também devem possuir datas definida para inicio e fim; limites de recursos e custos; bem como um quantitativo de pessoas e equipamentos que será envolvido nestas.

  Finalmente, de acordo com o \citeonline{pmiguide2014}, um projeto pode ser compreendido por um esforço temporário empreendido a fim de criar um produto, prestar serviço ou trazer um resultado exclusivo. Esse esforço é composto por um conjunto de atividades inter relacionadas e direcionadas à obtenção de um ou mais produtos únicos, com tempo e custos definidos. Nesta definição, algumas características básicas do projeto devem ser destacadas:

  \begin{itemize}
    \item{\textbf{Delimitação temporal:} datas específicas para início e fim;}
    \item{\textbf{Objetivos:} metas definidas em função de um problema, oportunidade ou interesse da organização;}
    \item{\textbf{Elaboração progressiva:} etapas contínuas de desenvolvimento e incrementos;}
    \item{\textbf{Incerteza:} a representação do degrau entre o resultado esperado e as condições de realização do projeto;}
    \item{\textbf{Singularidade:} a representação da exclusividade do projeto, aquilo que o torna único;}
    \item{\textbf{Relação fornecedor-beneficiário:} relação entre quem desenvolve e quem recebe o projeto.}
  \end{itemize}

  Assim, pode-se compreender que todo projeto é essencialmente temporário e único, ou ainda, finito e regular, que frequentemente é utilizado, direta ou indiretamente, para alcançar os objetivos de um plano estratégico que, por sua vez, visa o desenvolvimento de um novo produto ou um serviço exclusivo.

\subsection{Gestão de Projetos}

  De acordo com \citeonline[p. 74]{kerzner2013project} a gestão de projetos pode ser considerada uma metodologia que consiste em um processo repetitivo usado em projetos com o objetivo de alcançar sua maturidade. Afirma-se também que qualquer metodologia, inclusive a mais simples, pode representar um caso de sucesso como prática de GP, desde que seja aceita na organização em questão. Entretanto, ao utilizar uma metodologia de GP de sucesso, a probabilidade de que a organização se destaque como entregadora de bons projetos será elevada \cite{kerzner2013project}.

  Para \citeonline{pmiguide2014}, a GP implica no uso de ferramentas, técnicas e da competência de utilizar recursos como: o conhecimento de conceitos, características próprias e particulares, bem como fatores críticos de sucesso para o aprimoramento e entrega de projetos de excelência. O guia PMBOK é considerado um conjunto das melhores práticas de GP, que se encontra dividida em cinco grupos de processos e dez áreas de conhecimento \cite{pmiguide2014}.

  São os grupos de processos:
  \singlespacing
  \begin{enumerate}
    \item Iniciação;
    \item Planejamento;
    \item Execução;
    \item Monitoramento e Controle;
    \item Encerramento.
  \end{enumerate}
  \onehalfspacing

  Quanto as áreas de conhecimento, constam os gerenciamentos de:

  \singlespacing
  \begin{enumerate}
    \item Integração;
    \item Escopo;
    \item Tempo;
    \item Custos;
    \item Qualidade;
    \item Recursos Humanos;
    \item Riscos;
    \item Comunicação;
    \item Aquisições;
    \item Partes Interessadas.
  \end{enumerate}
  \onehalfspacing


  A Figura~\ref{processos_areas_pmbok} representa a divisão da relação das àreas de conhecimento pelos grupos de processos.

  \begin{figure}[ht]
    \centering
    \scalebox{0.3}{\includegraphics{figuras/processos_areas_pmbok}}
    \caption{Relação dos grupos de processos pelas àreas de conhecimento. Fonte: \cite{pmiguide2014}}
    \label{processos_areas_pmbok}
  \end{figure}

  Alguns autores afirmam que a garantia de que os objetivos definidos de projeto serão alcançados depende de um processo disciplinado, por parte da GP, que respeite custos, prazos e desempenho requeridos e que ocorra através do envolvimento de pessoas em atividades de planejamento e controle numa organização \cite{dinsmore2009ama, meredith2011project}.

\subsection{Gestão de Programas}

  Programas podem ser entendidos por estruturas que consistem em uma equipe principal e um conjunto de equipes de projeto que averiguam capacidade de decisão e autoridade de um membro definitivo, isto é, um gerente de programa que assegura a direção e as decisões desta estrutura. Estas estruturas visam alcançar um determinado objetivo dentro de uma estratégia.\cite{brown2008handbook}.

  \citeonline{rijke20141197} avalia que apesar da dificuldade geral em distinguir um programa de um projeto, a gestão de programas deve ser considerada mais extensa que gestão de projetos, pois ela abrange áreas em que projetos singulares não se encontram. Vale ressaltar também que o gestor de programa tem hábitos mais estratégicos que podem interferir na GP \cite{lycett2004289}.

  Assim, a gestão de programas tem sido cada vez mais adotada por organizações com o objetivo de implementar estratégias que integrem melhor seus projetos e ferramentas, sem permitir que o desempenho possa desorientar a natureza estratégica das decisões.

\subsection{Gestão de Portfólio}

  A gestão de portfólio, conhecida também por Gestão de Portfólio de Projetos (GPP), surgiu a partir da necessidade de gerenciar investimentos em projetos nas organizações. Seu processo dinâmico e integrado visa avaliar o alinhamento estratégico e a viabilidade da execução simultânea de diversos projetos ao mesmo tempo. Esses projetos passam por uma introspecção que os seleciona e organiza de acordo com sua priorização em um portfólio \cite{meredith2011project, kerzner2013project}.

  De acordo com \citeonline[p. 27]{burke2013project}, um portfólio pode abrigar conjuntos de projetos, programas e até mesmo outros portfólios que não precisam estar diretamente relacionados, mas que se reúnem por uma questão de otimização e controle. A Figura~\ref{port_prog_proj} ilustra a relação entre portfólio, programas e projetos.

  \begin{figure}[ht]
    \centering
    \scalebox{0.4}{\includegraphics{figuras/port-prog-proj}}
    \caption{Relação Portfólio, Programas e Projetos. Fonte: \cite{pmi2006}}
    \label{port_prog_proj}
  \end{figure}

  \citeonline[p 11]{pmiguide2014} estabelece que o critério de agrupamento de um portfólio deve visar a facilitação na gestão para que seja possível atingir os objetivos estratégicos de uma organização. É definido ainda que toda gestão de portfólio fique sob responsabilidade de um EGP. A figura \ref{estrategia_portfolio} representa os processos incorporados pela Gestão de Portfólio.

  \begin{figure}[ht]
    \centering
    \scalebox{0.4}{\includegraphics{figuras/estrategia-portfolio}}
    \caption{Processos de Gestão de Portfólio. Fonte: \cite{pmi2006}}
    \label{estrategia_portfolio}
  \end{figure}

  Desta forma, a GPP pode ser dita como uma manifestação de estratégia de negócios que determina os investimentos da organização via processos simultâneos, sistemáticos e dinâmicos de decisão, tornando o sucesso do EGP dependente do desempenho agregado de iniciativas dos componentes do portfólio e buscando a maximização do uso e do alinhamento desses componentes \cite{pmiguide2014}.

\subsection{Escritorio de Gestão de Projetos}

  Advindo do inglês \textit{Project Management Office}(PMO), o escritório de gestão de projetos (EGP), ou ainda escritório de projetos (EP), pode ser considerado um conjunto de profissionais de GP que servem a um modelo organizacional com o propósito de aumentar a eficiência e lidar com as necessidades da GP, assumindo um papel de alta confiança ao implementar diversas estratégias em projetos \cite{kendall2003advanced}.

  Através de um estudo, \citeonline{pemsel2013project} identificou três principais atividades que são esperadas do EGP:

  \begin{itemize}
    \item Promover e facilitar o desenvolvimento estratégico da GP, bem como o uso estratégico de objetos que sejam empreendidos na GP;
    \item Planejar, controlar e dar suporte a GP, sempre assegurando que o conhecimento seja compartilhado no processo para melhorar sua eficiência;
    \item Adoção estratégias de treinamento, negociação e formação para prover o desenvolvimento de competências.
  \end{itemize}

  Para \citeonline{dinsmore2005pmo} a principal expectativa empregada por um EGP esta relacionada ao suporte e orientação; ao processo de desenvolvimento e gerenciamento de projetos mais eficiente e eficaz o possível; e ao uso de metodologias e recursos de planejamento e análise de projetos padronizadas.

  De acordo com \citeonline{crawford2010strategic}, por mais simples que um EGP possa ser, suas atividades compõem estruturas complexas responsáveis por atividades de planejamento, controle e monitoramento, cuja implantação representa um processo de mudança de cultura organizacional, por estar diretamente relacionada à negociação com pessoas. O autor retrata ainda três níveis de atuação do EGP que podem ser visualizados na Figura~\ref{pmo_crawford}.

  \begin{figure}[ht]
    \centering
    \scalebox{0.6}{\includegraphics{figuras/pmo-crawford}}
    \caption{Níveis de atuação do escritório de projetos. Fonte: \cite{crawford2010strategic}.}
    \label{pmo_crawford}
  \end{figure}

  As competências desses níveis podem ser representas como \cite{crawford2010strategic}:

  \begin{itemize}
    \item \textbf{Nível 1 - Escritório de Controle de Projetos:} visa o desenvolvimento do planejamento de projetos individualmente, realizando também a emissão de relatórios de progresso. Embora tenha foco em apenas um projeto, geralmente este projetos apresenta grande porte e complexidade;
    \item \textbf{Nível 2 - Escritório de Projetos da Unidade de Negócios:} oferece suporte a todos os projetos de uma área específica, porém de porte e complexidade variados;
    \item \textbf{Nível 3 - Escritório Estratégico de Projetos:} possui as seguintes competências:
    \begin{itemize}
      \item Selecionar, priorizar e garantir a integração de cada projeto para que esteja alinhado à estratégia da organização, inclusive no que respeito aos seus recursos;
      \item Desenvolver, atualizar e divulgar a metodologia de GP bem como seus conhecimentos;
      \item Converte-se em centro da gestão de conhecimento da organização através do armazenamento de informações sobre lições aprendidas;
      \item Validar estimativas de recursos realizadas pelo projetos, baseando-se em experiências anteriores.
    \end{itemize}
  \end{itemize}

  Assim, as responsabilidades de um EGP podem variar de acordo com a centralização empregada na organização, uma vez que está relacionada a padronização dos processos de GP. Entretanto as ferramentas e técnicas a serem empregadas ficam sob critério do gestor responsável pelo EGP \cite{pmiguide2014}.

  Alguns autores, ainda, apontam o EGP como uma ferramenta de apoio para que organizações obtenham bom desempenho em GP, bom como para alcançarem seus objetivos estratégicos. Esses mesmos autores destacam que fatores comuns ligados as taxas de sucesso do EGP, em caso positivo, devem ser enfatizados, enquanto em caso negativos, devem ser evitados, configurando assim boas práticas para o sucesso do EGP \cite{andersen2007benchmarking}.

\subsection{Sistemas de Gestão de Projeto}

  Ao contrário do que é esperado pela natureza de negócios ou operações usuais, a natureza dos projetos e serviços é representada por processos curtos, repetitivos e funcionais, que facilitam a identificação de padrões usualmente inseridos em soluções de informação. Assim, especialista de GP veem o uso de sistemas de gestão de produtos (SGP), como uma ferramenta preciosa no que respeito a alcanço os objetivos e na excelência de projetos \cite{cserban2011project}.

  Seguindo mais além, \citeonline{prado2006mmgp} afirma que diversos aspectos das metologias de GP dependem indiretamente do uso de SGP, sem que porém, seja determinada a natureza do SGP. O autor infere ainda que só é possível para organizações almejarem determinados níveis de maturidade se utilizarem essas ferramentas.

  Através de um estudo quantitativo, \citeonline{liberatore2001project} analisou que nunca antes a adoção de uma ferramenta para auxiliar nas práticas de GP foi tão explorada quanto o uso de SGP. Os especialistas de GP demonstraram grande interesse pela facilidade encontrada no compreendimento da complexidade do projeto através desses sistemas, e expressaram interesse em integrar cada vez funcionalidades referente aos projetos.

  Por meio de uma revisão na literatura \citeonline{hartmann2009implementing}, acrescentou que além de auxiliar no tratamento da complexidade nos projetos, o uso de SGP também se destaca para o aprimoramento da produtividade do processo de projeto, apesar de também ser notada uma dificuldade inicial para o adaptamento do uso dessas ferramentas.

  Além de prover um meio para lidar com a produtividade de processo e complexidade do projeto, já foi comprovado também que o uso de SGP influi nos processos de planejamento, comunicação, monitoramento, controle de riscos, cronograma, gerenciamento de documentos e ainda avaliação de custos \cite{raymond2008project}.

  Portanto, o uso de SGP implica em deter ferramentas capazes de facilitar e otimizar o esforço empregado pela GP para alcançar a excelência na realização do projeto, não apenas por parte do uso dos gestores do projeto, mas também por incluir outros atores presentes em seus processos \cite{cserban2011project}.

\section{Associações de Gestão de Projetos}

  \begin{quadro}[!htpb]
    \centering
    \label{quadro-institutos}
    \caption{Principais associações de Gestão de Projetos \\ Fonte: Adaptação de \citeonline{patah2012metodos}}
    \begin{tabular}{| m{.40\textwidth} m{.30\textwidth} m{.20\textwidth}|}
      \hline
      \textbf{Institutos} & \textbf{Conjunto de Métodos} & \textbf{País de Origem} \\
      \hline \hline
      \textit{Project Management Institute} (PMI) & \textit{Project Management Body of Knowledge} (PMBoK) & EUA \\ \hline
      \textit{International Project Management Association} (IPMA) & IPMA \textit{Competence Baseline} & União Européia \\ \hline
      \textit{Association for Project Management} (APM) & APM \textit{Body of Knowledge} & Reino Unido \\ \hline
    \end{tabular}
  \end{quadro}

  \subsection{Project Management Institute (PMI)}

  O PMI foi fundado em 1969, como uma instituição sem fins lucrativos, que se baseou na premissa de que existiam muitas práticas comuns de GP em diversas áreas de projetos, entre elas, nas áreas de contrução e de produção farmaceútica (PMI, 2008; PMI, 2010a). Atualmente esse instituto é conhecido por disseminar o conhecimento sobre GP em diversos paises, sendo sua sede situada nos Estados Unidos (EUA), conforme ilustrado no Quadro~\ref{quadro-institutos}

O PMI foi fundado em 1969, é um dos institutos que atua no desenvolvimento e disseminação do conhecimento de Gestão de Projetos. Sua sede é nos Estados Unidos, e possui 240.000 membros distribuídos em mais de 160 países. Suas certificações e publicações são reconhecidas mundialmente, entre elas o PMBOK - Project Managemant Body of Knowledge. O PMI possui representações locais chamadas capítulos 1 . Prado e Archibald (2004, p. 17) afirmam que o PMI é quem padroniza e comanda a maioria das ações deste assunto em todo o mundo”.

De modo geral, o que se observa é que o PMBoK é um conjunto de métodos genérico e bastante abrangente que objetiva atender às necessidades dos mais diversos tipos de projetos (PMI, 2008).
O PMI (2009) apresenta um número de US\$ 12 trilhões, um quinto do valor do PIB mundial, como o valor a ser investido em projetos em cada um dos anos da atual década.

296.377 é o número de filiados ao PMI;11
o número de filiados cresceu 13,8\% entre fevereiro de 2008 e fevereiro de 2009;
existem 327.250 gerentes de projetos certificados PMP no mundo;
existem 7.207 profissionais certificados Certified Associate in Project Management (CAPM) no mundo;
o número de visitas (feitas apenas em janeiro e fevereiro deste ano) à website
oficial do PMI é de 2.444.645.

Conforme o PMI (2010a), ao longo de sua existência, o Instituto desenvolveu ferramentas que auxiliam o desenvolvimento de esforços de projetos. Uma dessas ferramentas, ou meios, de atualização é o complexo criado e mantido pelo mesmo, formado por publicações periódicas (PM Today, PM Journal, PM Network), publicações de padrões normativos (Project Management Body of Knowledge, The Standar for Portfólio Management, The Standar for Program Management, Government Extension for PMBOK Guide, Contruction Extension for PMBOK, entre outros), website, eventos regionais e globais (Global Congresses) e certificações (CAPM, PMP, PMI-SP, PgMP, PMI-RMP).
Aborda, como destacado em seu título para a publicação em português, “Um Guia do Conjunto de Conhecimentos em Gerenciamento de Projetos”, e é considerado bibliografia indispensável para o Gerenciamento de Projetos e para a Certificação PMP.

Da mesma forma, esse projeto trazia, como sugestão para desenvolvimento, as mesmas três áreas de concentração (ética, normas e credenciamento) (PMI, 2008). São elas:
y as características distintas de um profissional (ética);
y o conteúdo e a estrutura do conjunto de conhecimentos da profissão (normas);
y o reconhecimento de capacitação profissional (credenciamento).

Nesse projeto, o conteúdo original foi ampliado e reestruturado, agregando mais três
novas seções. São elas:
y inclusão da estrutura de Gerenciamento de Projetos para cobrir as relações entre o projeto e o seu ambiente externo, e também entre o Gerenciamento de Projetos e o gerenciamento geral;
y inclusão do gerenciamento de riscos como mais uma Área de Conhecimento;
y inclusão do gerenciamento de contratos/aquisições como mais uma Área de
Conhecimento.
Finalizando o ciclo com diversas mudanças e correções editoriais, em março de
1987, a Diretoria do PMI aprovou o manuscrito final como um documento
independente. Esse manuscrito foi publicado no mesmo ano com o nome Project
Management Body of Knowledge ou “O Conjunto de Conhecimento em
Gerenciamento de Projetos” (PMI, 2008).

O Guia PMBOK 5a edição, padrão do PMI para Gestão de Projetos, apresenta como melhores práticas, 47 processos de gerenciamento de projetos, distribuídos em 5 grupos de processos. Conforme o anexo 1, são eles: [1] iniciação, [2] planejamento, [3] execução, [ 4] monitoramento e controle e [5] encerramento. (PMI, 2013, p. 5).


  \subsection{International Project Management Association (IPMA)}

  \subsection{Association for Project Management (APM)}


\section{Maturidade em Gestão de Projetos}

Maturidade é o desenvolvimento de sistemas e processos que são por
natureza repetitivos e garantem uma alta probabilidade de que cada
um deles seja um sucesso. Entretanto, processos e sistemas repetitivos
não são, por si só, garantia de sucesso. Apenas aumentam a sua
probabilidade. Harold Kerzner

A maturidade é uma qualidade ou estado de amadurecimento. Se
aplicarmos o conceito de maturidade em uma organização, podemos relacionar a
maturidade com o estado no qual a organização está em perfeitas condições para
alcançar seus objetivos

Atualmente as organizações são avaliadas quanto ao nível de maturidade
dentro de uma escala definida por cada um dos modelos de maturidade em gestão de
projetos já propostos.

Os modelos de maturidade partem da premissa que as organizações,
pessoas e setores evoluem por meio de um processo contínuo de desenvolvimento ou
crescimento em direção a uma maturidade mais avançada. Os modelos têm por
finalidade auxiliar na elaboração de processos e execução de melhores práticas para
que as organizações se desenvolvam de forma constante.

> maturidade em gestao de projetos

Maturidade em Gestão de Projetos, ou maturidade organizacional em Gestão de Projetos é definido pelo PMI (2013, p. 552) como “o nível de habilidade de uma organização de entregar os resultados estratégicos desejados de maneira previsível, controlável e confiável”.
Em outras palavras, a maturidade em GP pode ser entendida como a nível de habilidade e conhecimento da organização para gerenciar seus projetos e obter sucesso, considerando que o sucesso está relacionado com a conclusão do projeto dentro do escopo, prazo, custo e qualidade planejado, satisfação do cliente, entre outros aspectos.
Kerzner (2000, apud RABECHINI JR., p. 33) apresenta um ciclo de vida da Gestão de Projetos, conforme a Figura~\ref{ciclo-maturidade}.

  \begin{figure}[ht]
    \centering
    \scalebox{0.6}{\includegraphics{figuras/ciclo-maturidade}}
    \caption{Ciclo de Vida da Gestão de Projetos.\\ Fonte: Adaptado de \cite{kerzner2006projeto}.}
    \label{ciclo-maturidade}
  \end{figure}

Os modelos de maturidade mais utilizados no Brasil são o OPM3 – Organizational Project Management Maturity Model, proposto pelo PMI, e o MMGP – Modelo de Maturidade em Gerenciamento de Projetos, proposto por Prado e Archibald. Conforme o PMSURVEY 10 do PMI (2014b, p. 42), este último é o modelo de maturidade mais utilizado entre nas instituições públicas brasileiras.

Os principais modelos existentes apresentam cinco níveis de maturidade, mas diferem um pouco quanto ao conteúdo de cada nível.
Prado e Archibald (2015) afirmam que quanto maior a maturidade, [1] maior o índice de sucesso total, [2] menor o índice médio de atraso, [3] menor o estouro médio de custos e [4] maior a confiança da alta administração em sua capacidade de planejar e executar projetos com sucesso.


> PMMM
> CMM-I???
> OPM3
  \subsubsection{Modelo de Maturidade em Gerenciamento de Projetos}

    \begin{figure}[ht]
      \centering
      \scalebox{0.6}{\includegraphics{figuras/mmgp}}
      \caption{Dimensões e níveis de maturidade do modelo Prado - MMGP.\\ Fonte: Prado e Archibald (2006, p.131).??? \cite{kerzner2006projeto}.}
      \label{mmgp}
    \end{figure}

\section{Fatores Críticos de Sucesso}
> conceito sucesso
Sucesso em Gestão de Projetos é frequentemente definido como a conclusão do projeto com atendimento da totalidade do escopo dentro do prazo e custo planejados e com a qualidade esperada. No entanto, existem variações deste entendimento.
Shenhar, Levy e Dvir (1997 apud NORO; BONZATTI, 2013, p. 86) afirmam que as pessoas possuem diferentes percepções em relação ao conceito de sucesso, e que essa percepção varia no tempo.
Kerzner (2006, p. 41-42) afirma que “o problema de definir sucesso como a concretização do prazo programado, dentro do orçamento e com níveis de qualidade desejado é que todos estes indicadores constituem uma definição interna de sucesso”. Segundo o autor o sucesso é definido pelo cliente. “Pode-se concluir um projeto internamente no prazo, no orçamento e nos limites de qualidade para só então descobrir que o cliente não gostou dos resultados.”

Vargas (2003, p. 18) afirma que “um projeto bem-sucedido é aquele que é realizado conforme planejado”. Como a maioria dos autores, Vargas (2003, p. 19) considera que o sucesso de um projeto pode ser medido pela obtenção dos resultados esperados, dentro do prazo, custo e qualidade desejados. Porém, ressalta que outros parâmetros são importantes. O autor define os seguintes critérios para considerar um projeto como bem-sucedido:
- Ser concluído dentro do tempo previsto;
- Ser concluído dentro do orçamento previsto;
- Ter utilizado recursos (materiais, equipamentos e pessoas) eficientemente, sem desperdícios;
- Ter atingido a qualidade e a performance desejada;
- Ter sido concluído com o mínimo possível de alteração de seu escopo;
- Ter sido aceito sem restrições pelo contratante ou cliente;
- Ter sido empreendido sem que ocorresse interrupção ou prejuízo nas atividades normais da organização;
- Não ter agredido a cultura da organização.

Prado e Archibald (2015) classificam os projetos em três categorias quanto ao sucesso:
Sucesso total: Um projeto bem-sucedido é aquele que atingiu a meta. Isto geralmente significa que foi concluído e produziu os resultados e benefícios esperados e os principais envolvidos ficaram plenamente satisfeitos. Além disso, mas não obrigatoriamente, espera-se que o projeto tenha sido encerrado dentro das exigências previstas para prazo, custo, escopo e qualidade (pequenas diferenças podem ser aceitas).
Sucesso parcial ou comprometido: o projeto foi concluído, mas não produziu todos os resultados e benefícios esperados. Existe uma significativa insatisfação entre os principais envolvidos. Além disso, provavelmente algumas das exigências previstas para prazo, custo, escopo e qualidade foram significativamente excedidas.
Fracasso: existe uma enorme insatisfação entre os principais envolvidos ou porque o projeto não foi concluído ou porque não atendeu às expectativas dos principais envolvidos ou porque algumas das exigências previstas para prazo, custo, escopo e qualidade foram excedidas de forma absolutamente inaceitável.

Conforme o PMI (2015, p. 8) a pesquisa Pulse revelou em média o percentual de projetos bem sucedidos é de 64\%, sendo considerado como projeto bem-sucedido aquele que atinge seus objetivos. O PMI (2015, p. 5) acredita que até que mais organizações comecem a investir na compreensão do valor da GP, no desenvolvimento de talentos (capacitação) na padronização de processos, as taxas gerais de sucesso de projetos não irão melhorar.

> fatores críticos de sucesso em GP


\chapter{A INOVAÇÃO A PARTIR DO MODELO UEG}
\thispagestyle{empty}

\section{O modelo Universidade-Empresa-Governo}

Para Sáenz e Capote (2002) ciência pode ser definida por um método, uma instituição, uma tradição acumulativa de conhecimentos, um fator fundamental na manutenção e desenvolvimento da produção, e uma poderosa influência que deve refletir sua complexidade em diferentes aspectos na formação de crenças e atitudes relativas ao universo.

Compreende-se também que os avanços na ciência sempre estarão ligados a mudanças nas forças produtivas, e que por assim dizer estão diretamente ligados a inovação.

Segundo Etzkowitz (2003), a inovação pode ser linear, reversa, assistida ou interativa, ou um processo segue uma ordem natural em que a pesquisa científica básica, aplicada ou tecnológica será disponibilizado no mercado.

Em um modelo linear reverso as demandas da sociedade servem de ponto de iniciação de processo, enquanto no modelo linear assistido existe o desenvolvimento de mecanismos de apoio para a intermediação das capacidades de transferência de tecnologia, ou mesmo calculo de capital de risco. Existe ainda um modelo interativo que incorpora as características dos demais modelos, atendendo simultaneamente a diversas demandas e criando processos de apoio a inovação.

Nas últimas décadas os governos vêm criando incentivos a criação de ambientes de inovação, como incubadoras e projetos de pesquisa, que atuam dentro das universidades, sendo liderados muitas vezes pelos próprios projetos e com participação de empresas em diversos momentos, desde a criação desses projetos, ao oferecimento de mão de obra e/ou bolsas que suportem as pesquisas.

Essa missão que está voltada a trazer inovação para comunidade com bem competitivo e de eficiência veio através de um modelo desenvolvido a anos atrás.

Por volta do final dos anos 90, Henry Etzkovitz desenvolveu um modelo de inovação que visava explorar a relação UEG, este modelo veio a ser conhecido por tríplice hélice.

Neste modelo, as múltiplas relações consideradas recíprocas, eram dívidas em diferentes estágios de processo e disseminação de conhecimento de forma espiral, cada esfera representa uma instituição independente que trabalha em cooperação com as demais esferas, através de fluxos de conhecimento (ETZKOWITZ, 1998).

Em 2000, Etzkowitz e Leydesdorff apresentaram uma visão da evolução dos SI, nas quais percebiam-se os conflitos potenciais nas relações de UEG. Portanto, essa nova visão trazia novas abordagens para essas variações nos arranjos institucionais dessas relações. A Figura 1 apresenta o modelo estático de relação UEG, onde o governo se envolve e passa a dirigir as relações entre as empresas e a Universidade.


\begin{figure}[ht]
    \centering
    \scalebox{0.7}{\includegraphics{figuras/ueg_estatico}}
    \caption{Modelo estático da relação UEG.}
    \label{crescimento_odf}
\end{figure}

Uma segunda abordagem mostrada pela Figura 2, representa as relações completamente separadas entre as partes em suas esferas institucionais, assim estabelecendo relações por base de independência entre as partes. Essa abordagem também é conhecida por modelo “laissez-faire” de relação UEG.


\begin{figure}[ht]
    \centering
    \scalebox{0.7}{\includegraphics{figuras/ueg_faire}}
    \caption{Modelo \'laissez-faire\' da relação UEG.}
    \label{crescimento_odf}
\end{figure}

Na terceira abordagem, o conceito de geração de conhecimento é estruturado através da sobreposição das esferas institucionais, e portanto sobrepondo as ações das partes para estabelecer as condições de desenvolvimento de uma verdadeira relação produtiva. O objetivo é promover a inovação do ambiente na busca pelo conhecimento, pelo desenvolvimento econômico e por alianças estratégicas com empresas.

Neste momento o papel do governo deixa de ser controlar as outras esferas, e passa a ser estimular parcerias. Existe ainda um espaço para colaboração trilateral na formação de organizações híbridas, conforme visto na Figura 3.


\begin{figure}[ht]
    \centering
    \scalebox{0.7}{\includegraphics{figuras/ueg_faire}}
    \caption{Modelo da Tríplice Hélice na relação UEG.}
    \label{crescimento_odf}
\end{figure}

Neste último modelo também, universidade deixa de ser uma instituição centrada basicamente no ensino e combina seus recursos e potenciais na área de pesquisa com uma nova missão, desenvolver ideias de caráter econômico e social da sociedade, e assim estimula o surgimento de ambientes de inovação e dissemina uma cultura empreendedora.

Podem ser definidos como quatro os processos relacionados com as mudanças baseadas no conhecimento pela tríplice hélice (ETZKOWITZ, 2003):

\begin{itemize}
  \item{mudanças internas em cada esfera, tais como estratégias de cooperação ou alianças entre empresas concorrentes, ou a incorporação do desenvolvimento econômico e social como missão da universidade e o papel de articulador para o governo;}
  \item{reconhecimento da influência de cada esfera nas ações dos demais, seja essa influência por meio de legislações governamentais nas áreas de propriedade intelectual, ou transferência de tecnologia e inovação (Lei Bayh-Dole nos Estados Unidos e Lei da Inovação no Brasil);}
  \item{criação de novos relacionamentos entre as partes, sejam alianças estratégicas ou outras formas de cooperação que estimulem a criatividade e a inter-relação regional, e ainda a criação de ambientes de inovação;}
  \item{a ampliação e repercussão das ações da relação UEG junto à sociedade.}
\end{itemize}

Para Etzkowitz (1998) as universidades, e por assim dizer, formas de ensino, passaram por duas grandes revoluções desde a sua criação, época em que eram centradas na transmissão de conhecimentos dos professores para os alunos, sem um movimento de troca, apenas de repassamento de informações.

Na primeira revolução, que se deu no final do século 17, o conceito de pesquisa é agregado como missão das universidades, além das atividades de ensino. A princípio notou-se que muitos desafios ainda viriam antes de uma total incorporação das atividades de pesquisa no meio acadêmico.

Na segunda metade do século 20, mesmo com o ideal de pesquisa ainda não amadurecido, veio a segundo revolução que trouxe o conceito de Universidade empreendedora. Além da missão de ensino e pesquisa, neste momento mais um valor foi agregado como missão, o valor de formação de conhecimento voltado também ao desenvolvimento econômico e social.

Com este acréscimo a missão das universidades, houve uma proximidade entre essas e a sociedade, que começou a se sentir verdadeiramente inserida neste meio.

Alguns autores, como Clark (2003), tratam a universidade empreendedora por Universidade inovadora, visto que a utilização de pesquisa e ciência permitem mudanças reais no mercado e portanto acarretam a inovação.

Assim, a concepção de universidade como simples “passadora” de conhecimento se modifica, se insere na sociedade, e passa a ser um dos principais papéis envolvidos na transformação de conhecimento gerado em valor econômico e social.

Entretanto, em 2003, Etzkowitz destaca que existem aspectos que prejudicam essa no época no ensino: na medida em que novos projetos surgem, conflitos de interesses também passar a surgir; esses conflitos de interesse podem ser decorrentes de interesses conflitantes, o não os ilegítima; assim a pesquisa e a comercialização dos resultados da pesquisa devem ser combinadas em um único modelo, visando evitar problemas.

Para Clark (2003) existem cinco fatores que endereçam questões críticas no processo de mudança nas universidades e portanto na relação UEG:

\begin{itemize}
  \item{a direção a seguir: no que respeito a estruturas gerenciais é preciso ter uma postura forte para que as mudanças sejam aceitas;}
  \item{o desenvolvimento expandido: é preciso estipular corretamente o desenvolvimento de novas estruturas frente as demandas da sociedade;}
  \item{as fontes de financiamento: diversificar as fontes de financiamento auxilia na ampliação dos recursos e portanto na estabilidade dos projetos;}
  \item{a estimulação das inovações: é preciso estimular os envolvidos nos processos de inovação para o processo seja bem aceito;}
  \item{a cultura empreendedora: é ideal criar uma cultura integrada, representada por uma visão compartilhada gerando uma perspectiva institucional.}
\end{itemize}

Assim, uma universidade empreendedora deve ser capaz de transformar pesquisa em resultados potenciais de comercialização, em ideias inovadoras tendo políticas de inovação como suporte de possibilidade de impacto regional.


\section{A relação UEG no Brasil}

No Brasil, durante os últimos 15 anos, tem existido um forte posicionamento em prol das atividades de PD\&I para inserção do meio acadêmico no dia a dia da sociedade frente as demandas da economia brasileira no mercado mundial e, desta forma, para estimular os SI, também entendidos como um conjunto de arranjos institucionais, cuja composição é dada pela relação UEG, levando à proposição de programas de incentivo à parceria (VEDOVELLO \& FIGUEIREDO, 2006; SBRAGIA, 2006).

Entretanto, para Toffler (1990), nossos líderes políticos insistem no mito de o mercado, como sociedade industrial, esta destinado a perpetuar indefinidamente, e tristemente, muitos dos educadores, que por vezes são considerados agentes de mudança, dão continuidade a este pensamento. Assim, provavelmente, quando o futuro chegar junto a inovação essa ideia provará estar apenas obsoleta.

Essa preposição reforça que os SI sigam uma abordagem mais interativa, e menos linear, como o processo de empreendedorismo já antes citado por Schumpeter (1966), que enfatiza um processo em que projetos podem ser apoiados por diferentes organizações, tanto privadas quanto governamentais.

Assim, podemos considerar que enquanto modelo de relação UEG, a tríplice hélice apresenta um arranjo organizacional bem evoluído por integrar a interação dessas três partes nos SI, e que serviu de ponto de partida para criação das Universidades Empreendedoras, como peças chaves para uma sociedade baseada no conhecimento (ETZKOWITZ, 2004; ETZKOWITZ \& KLOFSTEN, 2005).

No cenário internacional, diversos autores (ETZKOWITZ, 1998; ETZKOWITZ \& LEYDESDORFF, 2000; ETZKOWITZ \& ZHOU, 2007; LEYDESDORFF \& VAN DEN BESSELAAR, 1994) reconhecem o modelo da tríplice hélice como fator positivo no desenvolvimento regional e principal agente de inovação e transformação da ciência e tecnologia em crescimento econômico.

Eles reconhecem ainda que esse crescimento da inovação esta diretamente ligado a capacidade de interação UEG prevista no modelo, bem como ao processo de aceitação da sociedade as mudanças na estrutura econômica.

Em suma, muitos estudos empíricos internacionais utilizaram seus esforços para estudar as relações e interações das partes frente a seu envolvimento, muitas vezes falhando em documentar sobre os aspectos da inovação em relação a sociedade.

Tijssen (2006) e Welsh et al. (2008) estudaram sobre como eram estabelecidas as relações entre as universidades e as empresas. Campbell \& Guttel (2005), Landry et al. (2006), Mueller (2006) e Welsh (2008) estudaram como se desenvolvia o recebimento e reinvestimento do suporte, ou fomento, recebido pelas universidades.

Por fim Landry et al. (2006), Mueller (2006) e Shane (2004) sugeriram que a relação UEG pode afetar a performance e o desenvolvimento de ambas as partes de forma positiva, seja através da pesquisa, de patentes, ou desenvolvimento econômico.

Sbragia (2006) avalia que no início, havia uma desconfiança mútua, fosse pela diferença de linguagem ou pelo choque de cultura, que resultava da falta de alinhamento entre as ideias e as imposições feitas pela pesquisa.

Em 2003, Dagnino \& Gomes levantou um caso exemplar de relação UEG realizado pela Faculdade de Mecânica da Universidade Estadual de Campinas (FEM/Unicamp) com a então multinacional de autopeças Clark Equipaments, inicialmente a proposta de interação veio através da realização de uma auditoria, posteriormente esta empresa se desenvolveu e se especializou de tal forma que veio a se desligar da filial se tornando Eaton Truck Corporation e vindo a firmar novos projetos tecnológicos com a participação da faculdade.

Através de um estudo que avaliou os dados do Diretório de pesquisa do CNPq no Censo 2002, Rapini (2006) destacou que existe a predominância de fluxos de conhecimento e serviços vindos de grupos de pesquisa para empresas, sendo esses serviços geralmente voltados a tarefas rotineiras de pouca complexidade. A autora destaca também que existe pouco uso do escopo de indicadores de CT\&I.

Ipiranga \& Almeida (2012), avaliaram a relação UEG através da Rede Nordeste Biotecnologia (Renorbio) e notaram que seu processo de cooperação era dotado de distinção de valores, linguagens, objetivos e cultura, o que tornava a interação entre as partes muito complexa, porém ativa para aqueles que estão diretamente envolvidos.

As autoras ainda destacam que um dos pontos que podem complexar a interação podem estar relacionados a falta de uma estrutura de gerenciamento, entretanto todos os resultados de pesquisa desenvolvidas estão patenteadas, demonstrando a viabilidade da inovação.

Berni et al.(2015) relataram sobre a experiência vivida na Universidade Federal de Santa Maria (UFSM) pelo processo de incubadoras. De modo geral a experiência demonstrou ser positiva quanto a interação UEG, sendo mais voltado para as partes de universidade e empresa.

Para a empresa citou-se que houve auxílio ao desenvolvimento de novos produtos, enquanto para a universidade, houve o auxílio a formação de profissionais e possibilidades de direcionamento para aplicações práticas e interações com a comunidade.

Alguns autores (FREEMAN, 1987; LUNDVALL, 1985; SUTZ, 1997; ETZKOWITZ, 2004) ressaltam que para alcançar o caminho da inovação é preciso cooperação de todas as partes, para que evoluam em um processo interativo e acumulativo que leva aos SI, sem cooperação não é possível compreender a distância entre o conhecimento e as diferentes  realidades da inovação.

Como Ivanova \& Leydesdorff (2014) previam, é importante que o incentivo para o vinculo da relação UEG venha, sobretudo, da implementação de políticas governamentais, que tenham por objetivo garantir estrategias em diversas áreas que busquem a excelência e potencial contribuição para o crescimento econômico., assim como a melhoria de condição de vida no contexto regional e mesmo de nação (LASTRES et al., 2005).


\chapter{ESTUDO DE CASO}
\thispagestyle{empty}

\section{Objeto de estudo}

Para Ristoff (2011) a Reforma Universitária garante que as universidades federais terão, enfim, a autonomia de gestão financeira prevista na constituição, mas nunca posta em prática; terão asseguradas a tão sonhada dotação global de recursos, a irredutibilidade nos repasses e a expansibilidade continuada.

Estarão, portanto, livres das amarras burocráticas e financeiras, que inibem a autogestão, repelem a inovação e forçam a privatização do espaço público. Ao defender a autonomia das universidades, nos termos do artigo 207 da Constituição, o governo também deixa claro o seu entendimento de que nestas instituições, embora não necessariamente nos Centros Universitários e Faculdades, as atividades de ensino, pesquisa e extensão são definidoras de sua natureza.

Em 2008, foram fundamentados os Institutos Federais (IF) no país, para que fosse possível dar mais um passo no processo de expansão da Rede Federal de Educação, e também como parte dos objetivos de Plano de Desenvolvimento da Educação (PDE) (BRASIL, 2016).

Um dos incentivos governamentais para conciliação de universidade, ou institutos, com empresas veio através da formalização da Empresa Brasileira de Pesquisa e Inovação Industrial (EMBRAPII) em 2013, para fomentar o processo de cooperação entre pequenas  e medias empresas nacionais e instituições tecnológicas ou privadas sem fins lucrativos.

Os primeiros projetos pilotos envolvem o Instituto de Pesquisa Tecnológico (IPT) na área de nanobiotecnologia, o Instituto Nacional de Tecnologia (INT) em energia (gás/petróleo) e saúde e o Centro Integrado de Manufatura e Tecnologia (CIMATEC) do Serviço Nacional de Aprendizagem Industrial (SENAI) na área de automação manufatura (EMBRAPII1, 2016).

Em março de 2015, a EMBRAPII selecionou cinco IF para atuarem em projetos de inovação industrial, para passarem pelo processo de seleção foi exigido das IF que participassem de um curso de capacitação de seus agentes de inovação, para que estivessem preparados para interagir com a relação UEG visando proporcionar eficiência  e agilidade no processo de transmissão de tecnologia para sociedade (EMBRAPII1, 2016).

Entre os polos selecionados destacasse o Polo de Inovação Campos dos Goytacazes pertencente ao Instituto Federal Fluminense (IFF) em parceria com o Campus Rio Paraíba do Sul (UPEA), inicialmente Unidade de Pesquisa e Extensão Agroambiental.

O polo está  localizado no município de Campos dos Goytacazes – RJ, no norte do estado do Rio de Janeiro, e foi reconhecido pelo Ministério da Educação em 13 de agosto de 2015 (PICG/IFFluminense – Portaria 819/2015). Desde sua inauguração a UPEA vem  realizando diversos trabalhos de fundamento ambiental para o atendimento das demandas regionais (EMBRAPII2, 2016).

Neste momento o PICG está sendo estruturado para desenvolver projeto de PD\&I e receberá um financiamento de R\$ 3 milhões para um plano de ação de 3 anos. Este plano de ação contará diretamente com a participação de empresas da região, visando transferir tecnologia para essas e, por fim, para a sociedade local (IFF, 2016).


\chapter{CONSIDERAÇÕES FINAIS}
\thispagestyle{empty}

Ao consultar o Portal de Inovação e Diretório de dados do CNPq, é possível notar que o Brasil possui inúmeros pesquisadores em diversas áreas do conhecimento, bem como diversos projetos de pesquisa, incubadoras e parques tecnológicos dignos de primeiro mundo.

Este trabalho apresentou uma leve perspectiva sobre um novo plano de ação do governo brasileiro envolvendo as novas IF, reformuladas e focadas em ensino de qualidade, e de Polos de Inovação, idealizados para reger as relações estabelecidas entre os institutos e as empresas locais onde estão localizados.

Para atender as expectativas de inovação e engajamento regional impostas ao PICG, este trabalho irá buscar modelar seus processos, com base na participação no escritório de projetos, utilizando os dos conceitos de PMO e GP já levantados, procurando sempre a melhoria e excelência nos processos.

Foi destacada também a importância notada pela relação UEG ao redor do país, e mais precisamente no desenvolvimento da relação do PICG, com as empresas regionais, para avaliar o impacto desta na sociedade e na transferência de tecnologia para empresas locais.

\newpage
\thispagestyle{empty}
\section{Cronograma}

\begin{enumerate}
  \item{Contato – Equipe Administrativa do Polo de Inovação Campos dos Goytacazes (PICG);}
  \item{Participação na criação do Escritório de Projetos do PICG;}
  \item{Revisão Bibliográfica;}
  \item{Apresentação do Projeto;}
  \item{Coleta de dados – escritório de projetos;}
  \item{Roteirização;}
  \item{Modelagem dos processos;}
  \item{Resultado da modelagem;}
  \item{Discussão sobre Aplicação da Modelagem;}
  \item{Aplicação da Modelagem;}
  \item{Discussão dos Resultados;}
  \item{Revisão do projeto final;}
  \item{Defesa.}
\end{enumerate}

\begin{table}[!htpb]
  \centering
  \begin{small}
    \setlength{\tabcolsep}{4pt}
    \begin{tabular}{|c|c|c|c|c|c|c|c|c|c|c|c|c|c|c|c|c|c|}\hline
     & \multicolumn{17}{c|}{Meses (2015 - 2016)}\\ \cline{2-18}
    \raisebox{1.5ex}{Etapa} & ago & set & out & nov & dez & jan & fev & mar & abr & mai & jun & jul & ago & set & out & nov & dez \\ \hline
    1 & X &   &   &   &   &   &   &   &   &   &   &   &   &   &   &   & \\ \hline
    2 & X & X & X & X & X & X &   &   &   &   &   &   &   &   &   &   & \\ \hline
    3 &   &   & X & X & X & X & X & X & X & X & X & X &   &   &   &   & \\ \hline
    4 &   &   &   &   &   &   &   &   &   &   & X &   &   &   &   &   & \\ \hline
    5 &   &   &   &   &   &   &   &   & X & X &   &   &   &   &   &   & \\ \hline
    6 &   &   &   &   &   &   &   &   &   & X & X &   &   &   &   &   & \\ \hline
    7 &   &   &   &   &   &   &   &   &   &   & X & X &   &   &   &   & \\ \hline
    8 &   &   &   &   &   &   &   &   &   &   &   & X & X &   &   &   & \\ \hline
    9 &   &   &   &   &   &   &   &   &   &   &   & X & X &   &   &   & \\ \hline
    10 &   &   &   &   &   &   &   &   &   &   &   &   & X & X &   &   & \\ \hline
    11 &   &   &   &   &   &   &   &   &   &   &   &   &   & X & X &   & \\ \hline
    12 &   &   &   &   &   &   &   &   &   &   &   &   &   &   &   & X & \\ \hline
    13 &   &   &   &   &   &   &   &   &   &   &   &   &   &   &   &   & X \\ \hline
    \end{tabular}
  \end{small}
  \caption{Cronograma das atividades previstas}
  \label{t_cronograma}
\end{table}


\include{cap6}
\bibliography{referencia} % Gera as referências bibliográficas
\apendice
\chapter{ANÁLISE BIBLIOMÉTRICA}

A bibliometria foi realizada por meio do levantamento bibliográfico relacionado ao tema da pesquisa, utilizando o mecanismo de busca Google Acadêmico, e as bases de periódicos da CAPES, \textit{SCOPUS} e \textit{ScienceDirect}. As principais expressões de busca utilizadas foram: ``project management'' , ``project management agile'', ``agile'' e a tradução destas para a língua portuguesa. A seguir será apresentado os passos desta análise.

\section{Material não disponibilizado pela base CAPES}

O levantamento bibliográfico das fontes impressas e digitais, que não se encontravam disponiveis pela Portal Capes, foi realizado através do portal \textit{ScienceDirect} e do mecanismo Google Acadêmico, que dispõe de bancos de teses e dissertações de instituições específicas, e aprensenta redirecionamento às bases Biblioteca Digital Brasileira de Teses e Dissertações (BDTD), Scientific Electronic Library Online (SciELO).

Ressalta-se ainda que o mecanismo de busca Google Acadêmico também indexa em suas pesquisas as bases de periódicos da CAPES, \textit{SCOPUS} e \textit{ScienceDirect}, permitindo ao usuário avaliar a relevância de citações de um determinado periódico em ambas as bases, e redirecionando o usuário as bases para leitura dos períodicos.

Foi realizada também, uma pesquisa sobre o tema para saber sobre o interesse no mesmo nos ultímos anos, para tanto foi utilizada a ferramenta Google \textit{Trends}. Primeiramente foi utilizada a frase de pesquisa ``project management'' para demonstrar o interesse específico sobre as práticas de GP. O resultado apresentado no gráfico \ref{trends1}, que parte do ano de 2006, declinando em pesquisas até o ano de 2016.


\begin{figure}[!ht]
  \centering
  \scalebox{0.5}{\includegraphics{figuras/trends1}}
  \caption{Interesse sobre práticas de GP ao passar do tempo. Fonte: Google \textit{Trends} (2016).}
  \label{trends1}
\end{figure}


Para saber o interesse sobre práticas de GP que fossem consideradas agéis foi somada a pesquisa a frase ``project management agile'', para que assim pudesse haver um comparação sobre o interesse dessas práticas com o passar do tempo. O resultado desta pesquisa é apontado no gráfico \ref{trends2}, onde o interesse pela gestão de projetos apresenta um declinío lento, enquanto o interesse pela gestão ágil de projetos cresce, também lentamente.

\begin{figure}[!ht]
  \centering
  \scalebox{0.5}{\includegraphics{figuras/trends2}}
  \caption{Interesse sobre práticas de GP versus práticas agéis ao passar do tempo. Fonte: Google \textit{Trends} (2016).}
  \label{trends2}
\end{figure}


Os indicadores bibliométricos escolhidos foram os seguintes: ano de publicação, idiomas, autores e tipos de documento. Seguindo os indicadores escolhidos, a seleção dos artigos foi feita considerando os seguintes critérios:

\begin{itemize}
  \item \textbf{Ano de Publicação:} foi considerado o periodo de 2005 a 2015, isto é, a equivalências aos ultimos dez anos, pois entende-se que este periodo compreende os anos de maior interesse no tema de pesquisa de acordo com mecanismo utilizado.
  \item \textbf{Idiomas:} foi dada preferência ao inglês pela importancia que os períodicos desse idioma representaram, entretanto também foram escolhidos períodicos em português pela sua relevância ao tema proposto.
  \item \textbf{Autores:} foram consideradas principalmente publicações de autores mais pesquisados e mais citados.
  \item \textbf{Tipos de Documento:} foi dada preferência a publicações em periodicos, por traduzirem pensamentos mais recentes relativos ao tema de pesquisa.
\end{itemize}


Os trabalhos selecionados a partir da análise bibliometrica, após a aplicação dos indicadores mencionados, encontram-se relacionados no quadro \ref{tabela_autores}.

\begin{longtable}{| p{.20\textwidth} | p{.80\textwidth} |}
  \cline{1-2}
  \cellcolor[HTML]{C0C0C0}{\color[HTML]{000000} Autor(es)} & \cellcolor[HTML]{C0C0C0}{\color[HTML]{000000} Publicação} \\ \hline
  ALMEIDA, L. F. M. et al. & Fatores críticos da agilidade no gerenciamento de projetos de desenvolvimento de novos produtos (2012) \\ \hline
  AMARAL, D. C. et al. & Gerenciamento ágil de projetos: aplicação em produtos inovadores (2011). \\ \hline
  ATKINSON, R. D.; EZELL, S. J.; STEWART, L. A. & The global innovation policy index(2012). \\ \hline
  BARLOW, J. B. et al. & Overview and guidance on agile development in large organizations(2011). \\ \hline
  BATRA, D. et al. & Balancing agile and structured development approaches to successfully manage large distributed software projects: A case study from the cruise line industry(2010). \\ \hline
  BECK, K. et al. & Manifesto for agile software development(2001). \\ \hline
  BERGGREN, C.; SÖDERLUND, J. & Rethinking project management education: Social twists and knowledge co-production(2008). \\ \hline
  BINDER, J.; AILLAUD, L. I.; SCHILLI, L. & The project management cocktail model: An approach for balancing agile and iso 21500(2014). \\ \hline
  BOEHM, B.; TURNER, R. & Management challenges to implementing agile processes in traditional development organizations(2005). \\ \hline
  CAMPBELL, D. F.; GUTTEL, W. H. & Knowledge production of firms: research networks and the"scientification"of business r\&d(2005). \\ \hline
  CICMIL, S. J. et al. & Exploring the complexity of projects: Implications of complexity theory for project management practice(2009). \\ \hline
  COCKBURN, A. & Agile software development(2006). \\ \hline
  COLLYER, S. et al. & Aim, fire, aim—project planning styles in dynamic environments(2010). \\ \hline
  CONFORTO, E. C.; AMARAL, D. C. & Evaluating an agile method for planning and controlling innovative projects(2010). \\ \hline
  ETZKOWITZ, H.; KLOFSTEN, M. & The innovating region: toward a theory of knowledge-based regional development(2005). \\ \hline
  FERNANDEZ, D. J.; FERNANDEZ, J. D. & Agile project management—agilism versus traditional approaches(2008). \\ \hline
  GANGULY, A.; NILCHIANI, R.; FARR, J. V. & Evaluating agility in corporate enterprises(2009). \\ \hline
  GERALDI, J. G. et al. & Innovation in project management: Voices of researchers(2008). \\ \hline
  HASS, K. B. & The blending of traditional and agile project management(2007). \\ \hline
  HIGHSMITH, J.; COCKBURN, A. & Agile software development: The business of innovation(2009). \\ \hline
  KOLLTVEIT, B. J.; KARLSEN, J. T.; GRØNHAUG, K. & Perspectives on project management(2007). \\ \hline
  LASTRES, H. M.; CASSIOLATO, J. E.; ARROIO, A. & Conhecimento, sistemas de inovação e desenvolvimento(2005). \\ \hline
  LEYBOURNE, S. A. & Improvisation and agile project management: a comparative consideration(2009). \\ \hline
  MIR, F. A.; PINNINGTON, A. H. & Exploring the value of project management: linking project management performance and project success(2014). \\ \hline
  QUMER, A.; HENDERSON-SELLERS, B. & An evaluation of the degree of agility in six agile methods and its applicability for method engineering(2010). \\ \hline
  STARE, A. & Agile project management in product development projects(2014). \\ \hline
  STYHRE, A. & The bureaucratization of the project manager function: The case of the construction industry(2006). \\ \hline
  WILLIAMS, T. & Assessing and moving on from the dominant project management discourse in the light of project overruns(2005). \\ \hline
  WINTER, M. et al. & Directions for future research in project management: The main findings of a uk government-funded research network(2006). \\ \hline
  \caption{Publicações Selecionadas. Autoria própria (2016).}
  \label{tabela_autores}
\end{longtable}

\include{anexos}
\end{document} % Fim do TCC
