\chapter{METODOLOGIA}
\thispagestyle{empty}

O projeto presente tem por finalidade analisar a efetividade da gestão de projetos em um escritório de projetos de um polo de inovação identificando a maturidade destes projetos através de um modelo de maturidade proposta a EMBRAPII. Neste contexto, este capítulo descreverá sobre a metodologia a ser utilizada.


\section{A natureza de pesquisa}

Esta pesquisa pode ser caracterizada como uma pesquisa bibliográfica, cujo método utilizado foi o estudo de caso. Visando atingir o objetivo da pesquisa, inicialmente, foi realizada uma pesquisa bibliográfica através do material existente na literatura.

Através da leitura foi possível identificar os elementos de pesquisa propostos por \citeonline{lakatos2010fundamentos}, são eles: intenção, reflexão, espírito crítico, atenção, análise e síntese.

Se utilizada a proposta de \citeonline{gil2002elaborar}, a presente pesquisa é tomada por qualitativa quanto a abordagem da problemática, e descritiva quanto a caracterização de seus objetivos, pois utiliza do desenvolvimento de um estudo de caso.

\citeonline[p 239]{fortin2009fundamentos} confirma que todo estudo que objetiva a identificação de características referentes a um fenômeno, visando obter uma visão geral de sua situação ou população em questão, pode ser apresentado como um estudo descritivo. Nestas condições, determinado trabalho consistirá na pretenção de descrever ou interpretar a propriedades desta investigação, através de análises empíricas e da descrição da problemática \cite{fortin2009fundamentos, lakatos2010fundamentos}.

\section{Método de Pesquisa}

Para \citeonline{de2007metodologia} todo e qualquer trabalho científico deve proporcionar a reprodução de experiências de modo que outros pesquisadores sejam capazes de obter resultados descritivos, de repetir suas observações e ainda de realizar julgamentos as conclusões de seu(s) autor(es).

De acordo com \citeonline{vergara2009projetos} pesquisas também podem ser caracterizadas em relação aos aspectos relativos aos fins e aos meios. A presente pesquisa apresenta influência no caráter exploratório descritivo quanto aos fins, pois teve como objetivo \'analisar a efetividade da gestão de projetos em um escritório de projetos de um polo de inovação \'e se propõe a expor características de uma determinada população, sem o compromisso de explicar os fenômenos que descreve, embora sirva de base para essa explicação.

Quanto aos meios, a pesquisa classifica-se como bibliográfica e de campo.

Assim, porque é de pretensão deste trabalho descrever e interpretar, mas do que avaliar um caso real presente, e seguindo o caráter de investigação revelado por \citeonline{fortin2009fundamentos}, este trabalho se apresenta como descritivo e exploratório.

Por sua vez, quanto aos procedimentos técnicos que foram utilizados para obtenção dos dados, seguindo a classificação proposta por \citeonline{lakatos2010fundamentos}, trata-se de pesquisa bibliográfica, que também pretende utilizar de uma pesquisa de campo.Uma pesquisa de campo é aquela que busca alcançar informações referentes a um problema e busca uma resposta ou de uma hipótese, que se pretende comprovar, bem como descobrir novos fenômenos ou as relações entre eles \cite{de2007metodologia}.


\section{População e Amostragem}

\citeonline[p 20]{de2011elaboraccao} define população pesquisa como aqueles a que ela se refere ou representa, isto é, o universo compreendido dentro da pesquisa. Nesta pesquisa, o universo abordado se refere aos funcionários envolvidos no processo de construção do escritório de projetos do PICG, dentro do contexto de gestão de projetos.

Segundo \citeonline{embrapii2013}, a Associação Brasileira de Pesquisa e Inovação Industrial (EMBRAPII) é qualificada como uma Organização Social, fundada em setembro de 2013, sob reconhecimento da capacidade de inovação brasileira via exploração da sinergia entre instituições de pesquisa tecnológica e empresas industriais. Sua missão é contribuir para o desenvolvimento da inovação na indústria através do fortalecimento de instuições de pesquisa tecnológicas, em selecionadas áreas de competência.

O Polo de Inovação Campos dos Goytacazes (PICG), conhecido até então por Unidade de Pesquisa e Extensão Agroambiental (UPEA) no momento em que foi inaugurado em outubro de 2007, foi criado no intuito de possibilitar o desenvolvimento de atividades de Pesquisa, Desenvolvimento e Inovação (P\&DI) associadas ao Instituto Federal Fluminense (IFF). Desde sua criação, a unidade realiza atividades de PD\&I, notadamente na área ambiental, em atendimento de demandas regionais, a partir de parcerias com empresas  órgãos de governo. Assim, em agosto de 2015, o PICG foi oficialmente reconhecido por integrar uma unidade EMBRAPII de Monitoramento e Intrumentação para o Ambiente \cite{embrapiiff}.

Este polo é também responsável pela formação de recursos humanos, via programa de capacitação desenvolvido na sede do Centro de Referência para profissionais de empresas parceiras, servidores, estudantes e responsáveis pela execução dos projetos. Além da capacitação técnica, o programa se responsabiliza pela gestão atual e a longo prazo, ampliando, dessa forma, as possibilidades do Polo de atender as solicitações das indústrias.

No momento de parceria entre a EMBRAPII e o PICG, foi estabelecido um modelo de financiamento que prevê ao polo autonomia de atuação, sob responsabilidade exclusiva de execução dos projetos, aplicação dos recursos financeiros e de prestação de contas. Neste modelo também foi estabelecido que os resultados, ou entregas, dos projetos seriam qualificados quanto ao nível de maturidade tecnológica do modelo EMBRAPII, sendo sempre necessário que os níveis estivessem considerados entre três e seis desta escala \cite{iffembrapii2016}.

Para atender a discussão científica proposta no presente estudo, de acordo com a posição de \citeonline{vergara2009projetos}, foi apresentada uma problemática referente ao universo que será exposta no próximo capítulo.

% \section{Limitações dos Métodos de Pesquisa}
