\anexo
\chapter{ARTIGO}

\begin{center}
{\large \textbf{Do tradicional ao ágil: uma perspectiva do gerenciamento de projetos}}
\end{center}

\begin{center}
\textbf{RESUMO}
\end{center}

\singlespacing
\noindent A chegada da era da inovação no mercado impôs diversas mudanças ao mundo, entre elas às práticas de gerenciamento de projetos, ou a sua versão tradicional, como ficou conhecida. Para lidar com a necessidade constante de modificações ao longo de projetos novas abordagens foram pautadas, de Gerenciamento Ágil de Projetos (GAP). Entretanto, a existência de estudos que comprovem a total eficácia do uso dessas abordagens ainda é escassa. Este artigo apresenta uma revisão bibliográfica de estudos que visaram divulgar essas abordagens, apontando vantagens e desvantagens de seu uso, e ainda de trabalhos que utilizaram técnicas de ambas abordagens, propondo novos modelos de seu uso. É possível argumentar que muitos projetos ainda não utilizam nenhuma abordagem, e que para os que utilizam abordagens tradicionais, a mesclagem com abordagens GAP são aconselháveis. \\
\noindent Palavras-chave: gestão de projetos, gestão agil de projetos, práticas GAP.


\begin{center}
\textbf{ABSTRACT}
\end{center}

\singlespacing
\noindent The arrival of the era of innovation in the market requires several changes to the world, including the project management practices, or the traditional version, as it became known. To cope with the constant need to change along new approaches projects were guided in Agile Project Management (APM). However, there are studies that prove the overall effectiveness of the use of these approaches is still scarce. This article presents a literature review of studies that aimed to disseminate these approaches, indicating advantages and disadvantages of their use, and still works using techniques of both approaches, proposing new models of its use. It is possible to argue that many projects still do not use any approach, and to those using tradicional approaches, merging with APM approaches are advisable. \\
\noindent Keywords: project management, agile project management, APM practices


\onehalfspacing

\section{INTRODUÇÃO}

Com a chegada do século 21, o mundo enfrentou diversas mudanças, tanto com respeito as inovações tecnológicas quanto ao cenário no mercado produtivo. Custo e qualidade deixaram de ser a principal motivação de melhoria nas empresas, cedendo espaço para projetos que entreguem produtos mais flexíveis em menor espaço de tempo (Yusuf 2004).

Neste contexto foi possível destacar a importância da inovação, do atendimento as constantes modificações de requisitos por parte dos clientes e ainda importância de manter o cliente como alvo da organização, para conquistá-lo e gerar vínculos de fidelização (CHIESA, 2007).

Na busca por soluções que tornassem possível a sobrevivência neste ambiente desafiador e imprevisível, a indústria decidiu voltar-se ao gerenciamento de projetos (GP), que no entanto, de acordo com Geraldi (2008), mostrou-se incapaz de lidar com tais mudanças a princípio.

Para Lin (2006) as ferramentas e técnicas tradicionais empregadas no GP não previam atividades de inovação nem instabilidades durante o processo de seus produtos, pois previam projetos linear e sem inconstâncias.

Visando atender a expectativa de mercado e a melhoraria da gestão de projetos, abordagens alternativas, que vinham se desenvolvendo desde o início dos anos 90, com novos princípios, ferramentas e técnicas, pareceram a melhor resposta. Essas abordagens vieram a ser conhecidas por Gerenciamento Ágil de Projetos (GAP) (Amaral 2011).

Entretanto é preciso que haja melhor compreensão sobre cada abordagem no que respeito aos processos executados na formação dos produtos em seu meio, e verificada adaptação destas abordagens.

Este artigo visa apresentar uma revisão bibliográfica sobre a transição das abordagens de GP, nas primeiras seções se apresentam as definições básicas de cada abordagem, seguido por uma seção de análise de estudos realizados na utilização dessas abordagens, e por fim as conclusões referentes a esta pesquisa.

\section{GERECIAMENTO DE PROJETOS}

Por definição, um projeto pode ser considerado um conjunto de atividades temporárias empregadas sobre recursos com a finalidade de entregar um serviço ou produto único (STELLINGWERF 2013). De acordo com o Instituto de Gerenciamento de Projetos (PMI, 2014), uma metodologia de GP deve conter um conjunto de métodos, técnicas, procedimentos, regras, modelos e boas práticas que possam ser usadas em projetos.

O uso dessas metodologias se tornou comum por volta do ano de 1950, e desde então vem se popularizando e tomando forma a medida que o mercado demonstra a necessidade de seu uso (Garel, 2013).

Para Eder (2012) a composição de uma ação, ou ato que gere resultados através do uso de um conjunto de técnicas e ferramentas, ou ainda um conjunto de artefatos que possam ser utilizados como ferramentas, descreve uma prática.

Dinsmore (2009) o GP deve ser realizado através de processos como iniciar, planeja, executar, controlar e encerrar, e deve contar com o uso de recursos como conhecimentos, habilidades, ferramentas e técnicas em atividades de projetos a fim de satisfazer seus requisitos.

Do ponto de vista de Charvat (2003), uma metodologia de GP representa um composto de orientações e princípios que podem ser adaptados e aplicados a situações específicas, onde este composto pode ser representado por uma simples lista de tarefas ou um conjunto de ferramentas e técnicas.

Para Shenhar (2007), o GP visa atingir tarefas e metas previstas em um plano estratégico de uma organização e serve de abordagem para a interação entre projetos, estratégia e objetivos organizacionais.

Similarmente, Cockburn (2000) define processo de software como um conjunto de ferramentas e práticas usadas para o desenvolvimento de produtos, e define GP como o conjunto de princípios que visa entregar um produto com sucesso.

Garel (2013) vê as metodologias de GP como conhecimento expresso em tarefas, técnicas, funções, ferramentas e entregas ao longo de um projeto em curso, que se utiliza de conhecimento sobre ajustes.

Em suma, todas metodologias de GP são similares enquanto se baseiam em abordagens, princípios e regras para satisfazer requisitos e entregar produtos de qualidade com sucesso.


\subsection{GERECIAMENTO DE PROJETOS TRADICIONAL}

Ao decorrer dos anos, as abordagens de GP foram disseminadas como “guias de conhecimento”, que apresentavam um conjunto de técnicas, princípios, ações e ferramentas que supostamente seriam capazes de gerenciar qualquer projeto (Kolltveit, 2007; Shenhar, 2007).

Inicialmente essas abordagens mantinham o foco no planejamento e controle do produto de forma engessada sem prever instabilidade e emergências, pois acreditava que era importante entregar o produto de acordo com os requisitos previamente levantados, indiferente a dissociações naturais do dia a dia (Winter, 2006).

Shenhar (2007) afirma que alguns autores investigaram sobre a correlação das práticas adotadas de acordo com o tipo de projeto descrito e das ferramentas utilizadas, e constataram que grande parte dos projetos eram geridos sem a devida diferenciação quanto a sua natureza, e com o uso de práticas genéricas denominadas “melhores práticas”.

De acordo com Collyer (2010) foi reconhecido pelo PMBOK que as práticas deveriam prever planos emergências que lidassem com conformidades não previstas nos requisitos, entretanto quanto a formação da ISO 21500, utilizada como padrão pela GP, nota-se que suas técnicas e ferramentas referiam-se com frequência ao modelo Cascata, quanto ao ciclo de vida do projeto apresentado por Boehm (1988).

Como explica Almeida (2012), é natural que práticas e abordagens evoluam ao longo dos tempos, como projetos passaram a ser o centro de toda e qualquer organização, entretanto, notou-se que a prática de GP se mantinha inadequada e que grande parte das falhas gerências estavam diretamente ligadas as técnicas utilizadas nestas praticas de GP.

A motivação das abordagens de GP consideradas tradicionais estava diretamente relacionada a perspectiva de que projetos eram relativamente simples, previsíveis eque seguiam de forma linear, tornando-os fáceis de detalhar e seguir ao longo do planejamento. Possibilitando assim, uma entrega eficiente dentro do prazo. (Andersen, 2006; Boehm, 2002; Boehm \& Turner, 2003; Cicmil, 2009; Collyer, 2010; Leffingwell, 2007; Shenhar, 2007; Williams, 2005; Wysocki, 2011).

Ainda, uma das vantagens descritas na prática tradicional era a capacidade robusta, isto é, a defesa de que diversos projetos poderiam sempre ser gerenciados seguindo as mesmas técnicas e ferramentas, como se fossem todos na mesma natureza.

Para diversos autores (Aguanno, 2004; Cicmil, 2009; Chin 2004; Shenhar, 2007; Williams, 2005; Wysocki, 2011) a robustez de planejamento é a maior desvantagem da prática tradicional, pois projetos têm se tornado cada vez mais complexo, e o para ideal de planejar linearmente não compete lidar com irregularidades dinâmicas da realizada do mercado.

Entretanto, Styhre (2006) destaca que no desejo por diminuir riscos e evitar ineficiências, o foco de gerenciar criatividade e mudanças é burocratizado e acaba por ser tornar um gerenciamento de papéis e formulários, geralmente encontrados na robustez do projeto.

Outra desvantagem notada é que por focar no planejamento e controle, essas abordagens ignoram não só a natureza do projeto, como também aspectos humanos quanto a sociabilização, contextualização e tendência a mudanças (Winter, 2006; Highsmith, 2009).

\subsection{GERENCIAMENTO DE PROJETOS ÁGIL}

Em função as desvantagens apontadas nas abordagens tradicionais de GP, somadas ao crescimento de um mercado inovador, com demandas contínuas e necessidade pela redução de custos, resultaram no advento de novas abordagens para GP (Aguanno, 2004; Conforto \& Amaral, 2010; Williams, 2005).

Em 2001, a partir de uma iniciativa em conjunto conhecida por Manifesto Ágil houve a formalização das abordagens ágeis. Seu diferencial se apresentava, simplificadamente, como um conjunto de técnicas, princípios e ferramentas que melhoravam o trabalho em time, permitindo adaptação e evoluções ao longo do projeto (Beck et al., 2001; Berggren, 2008; Cohn, 2005; Hass, 2007; Highsmith, 2009; Fernandez, 2008; Fitsilis, 2008; Larman, 2003; Salomo, 2007; Schwaber, 2004; Smith, 2007; Qumer, 2010).

De acordo com Cockburn (2000) e Schwaber (2001) existem quatro princípios ágeis que devem sempre ser enfatizados:
1. Indivíduos e Interações são mais importantes que processos e ferramentas;
2. Colaboração com as necessidades dos clientes acima de negociação de contratos;
3. Produto em funcionamento vale mais que documentação abrangente;
4. Responder as mudanças independente do plano a seguir.

Existem também práticas e ferramentas, como a concepção de visão do produto, desenvolvimento iterativo; além do uso de ferramentas visuais de planejamento do projeto e de seus artefatos (Augustine, 2005; Boehm \& Turner, 2004; Chin, 2004; Highsmith, 2009).

Ao ressaltar a importância dos itens, entretanto, os autores destacam que eles não se excluem, o ideal é que todos sejam levados em consideração, apenas seguindo a ordem de importância (Aguanno, 2004).

Essas abordagens podem ser apontadas por GAP, e representaram uma alternativa para problemática das mudanças constantes, alterações a curto prazo dentro do planejamento do projeto, valorização do cliente como uma fonte de interação (Amaral, 2011; Augustine, 2005; Cohn, 2005; Highsmith, 2009).

Alguns autores (Aguanno, 2004; Boehm, 1988; Manifesto, 2001; Williams, 2005), enfatizam que estas abordagens têm uma relação muito estreita com a engenharia de software, levando essas abordagens a aparecerem com frequência relacionadas ao desenvolvimento de software.

Stare (2014) afirma que durante uma entrevista os autores responsáveis pelo Manifesto Ágil ressaltaram a importância das abordagens GAP como um meio de tornar o GP uma ferramenta de sucesso no atual mercado, e que apesar de serem inicialmente aplicadas em projetos de tecnologia de informação (TI), eram igualmente apropriadas a qualquer natureza de projetos, visto que o foco destas abordagens está em reconhecer e aplicar feedback com a finalidade de lidar com incertezas.

Apesar desta declaração, Stare (2014) contesta que nos estudos pesquisados até o ano de 2009, em sua maioria, as práticas ágeis se referiam ao uso em projetos de TI, e que neste ano, algumas pesquisas começaram a reconhecê-las em outros campos com certo receio.

Entre as metodologias ágeis mais utilizadas, as que mais se destacam na indústria são: Scrum (Schwaber, 2004), Lean Software Development (Poppendieck, 2003), Crystal (Cockburn, 2004), Feature Driven Development (Palmer, 2002), Adaptive Software Development (Highsmith, 2001) e eXtreme Programming (Beck, 2000).

Considerando a crescente necessidade da indústria por inovação e pela necessidade de redução de custos, as abordagens GAP tomaram o cenário de GP devido sua habilidade de adaptação durante o ciclo de vida dos projetos, indiferente a sua natureza. Mudanças são inevitáveis e impossíveis de prever, portanto é importante lidar com elas (Aguanno, 2004; Conforto \& Amaral, 2010; Chin, 2004; DeCarlo, 2004; Highsmith, 2009; Leffingwell, 2007; Williams, 2005).

Além disso, como atrativos para utilização de práticas ágeis se destacam: a melhora na habilidade de comunicação formal e informal, bem como a colaboração externa com a aproximação dos clientes no processo de produtos (Aguanno, 2004; Cockburn, 2006; Collyer., 2010; Coram, 2005; DeCarlo, 2004; Highsmith, 2001; Williams, 2005).

Alguns autores destacam que a proximidade do time com o projeto melhora seu desempenho e a habilidade de ser autogerenciáveis e responsáveis em suas tarefas (Augustine, 2005; Boehm \& Turner, 2004; Highsmith, 2009).

Vale destacar que também existiram ideias de foram herdadas das práticas tradicionais, como o ideal de que o projeto fosse iterativo, isto é, pudesse ser incrementado ao longo do processo de produção (Boehm, 1988; Aguanno, 2004).

Entretanto, invés de utilizar um único plano para o projeto, GAP utilizam a ideia de iteratividade ao longo de todo processo através de pequenas fases com retorno geralmente fornecido por feedback (Boehm \& Turner, 2004; Highsmith, 2009; Schwaber, 2004; Augustine, 2005; Cohn, 2005).

Para Poppendieck (2003) o motivo pelo qual as práticas ágeis se destacaram esta diretamente relacionado a sua capacidade de entregar produtos com rapidez e qualidade, atendendo as necessidades do cliente de forma satisfatória utilizando princípios que já haviam sido citados pela metodologia Lean de produção.

A marca de agilidade é também propiciar uma melhoria de performance a medida que o cliente colabora mais ativamente frente a demonstração do produto; além de trazer o conceito de flexibilidade e estabilidade. (Chin, 2004; Dorairaj, 2012).

Para Aguanno (2004) uma das maiores vantagens do da GAP é a redução de riscos e falhas, a partir do memento que seu escopo pode ser definido ao longo do projeto, mudanças não se tornam transtornos e portanto não são erros.

Entre outras vantagens, é importante destacar o uso de ferramentas visuais que permitem organizar ideias, deixar claros objetivos e fases, tornar processos compreensíveis e facilitar o planejamento de projetos e gestão de portfólios, reduzindo riscos burocráticos (Malachia, 2013).

Alguns autores discutem que é preciso que as mudanças não sejam aplicadas apenas pela inovação do mercado, mas também na forma de pensar em GP, e consequentemente na estrutura da própria organização para que haja o correto emprego dos princípios e ferramentas observados (Aguanno, 2004; Boehm \& Turner, 2003; Chin, 2004; Cockburn, 2006; DeCarlo, 2004; Highsmith, 2009; Lawrence, 2006; Leffingwell, 2007; Shenhar, 2007).

Finalmente é preciso questionar se as práticas e princípios previstos nas abordagens GAP realmente se aplicam ao mercado, e se ao aplicá-los será possível entregar produtos de qualidade, com melhor custo e dentro do prazo.


\section{O USO DAS ABORDAGENS DE GERENCIAMENTO DE PROJETO}

Através de estudos, alguns autores (Coram, 2005; Conforto \& Amaral, 2010; Leybourne, 2009) notaram que alguns projetos procuram utilizar GAP como uma forma de fugir de ferramentas e técnicas particulares empregadas pelas abordagens tradicionais, e acreditam que estudos mais detalhados deveriam ser realizados nesses termos.

Para Cockburn (2000) toda abordagem tem seus limites, por mais bem embasada que possa ser, principalmente quando se tratam de produtos e clientes. Para que um determinado produto seja entregue com sucesso, é reconhecido que é ideal o uso de uma abordagem de GP, entretanto tanto a abordagem tradicional quanto a abordagem ágil possuem vantagens e desvantagens (Aguanno, 2004; Andersen, 2006).

Devido o aumento da demanda de inovação o crescimento do uso das abordagens, tanto individualmente quanto mescladas, tem sido o foco de estudos empíricos que buscam adaptá-las a diferentes tipos de projetos (Conforto \& Amaral, 2010).

De acordo com o estudo de Benassi (2011) existe evidencia de aspectos positivos no que diz respeito ao uso de GAP, entre elas: aumento na velocidade em que empresas alcançam inovações; cortes em custos excessivos, entre eles custos de armazenagem e desenvolvimento; diminuição do tempo de entrega do produto; reduz falhas no atendimento aos requisitos e portanto atende melhor ao desejo do cliente ao entregar produtos de qualidade em um prazo pequeno.

Entretanto, foi identificado também que a sugestão de pouca documentação em projetos é sugerida apenas para organização que possuem times pequenos e/ou bem organizados, que não detenham restrições corporativas e de procedimentos (Lindvall, 2004; Boehm \& Turner, 2003).

Alguns estudos definiram que projetos cujo o escopo é bem definido inicialmente, onde requisitos e metas possuem baixa nível de incerteza e mudança; e que podem ser desenvolvidos sem a presença frequente do cliente podem ser bem sucedidos com o uso de abordagens tradicionais, pois precisam focar apenas no planejamento e na otimização de atividades previstas (Boehm, 2002; Coram, 2005; DeCarlo, 2004; Fernandez, 2008; Shenhar, 2007; Wysocki, 2011).

Consequentemente, se o projeto for realizado em grandes corporações, independente de seu tamanho em particular, seu grau de complexidade ou sua duração; não é aconselhável que sejam utilizadas apenas abordagens GAP (Aguanno, 2004; Boehm, 2002; Boehm \& Turner, 2003; Cockburn, 2000; Fowler, 2005; Highsmith, 2009).

Para flexibilizar produtividade e satisfazer as necessidades de grandes organização, é sugerido que modelos sejam desenvolvidos, mesclando um pouco das ferramentas, técnicas e princípios de abordagens tradicionais e com GAP (Boehm \& Turner, 2003; Batra, 2010; Conforto \& Amaral, 2010; Barlow, 2011; Magdaleno, 2011).

Estudos recentes vêm utilizando referências teóricas e estudos empíricos para desenvolver modelos que sigam um pouco de cada abordagem (Lindvall, 2004; Batra , 2010; Barlow , 2011).

Mafakheri (2008) avalia o grau de agilidade de projeto, sob seis características: dinamismo; tamanho da equipe; comunicação; capacidade de testar resultados; conhecimento e habilidades relacionados ao produto. Não foi apresentado um estudo de caso comprovando a eficiência do modelo como uma prática de GP.

Qumer \& Henderson-Sellers (2008) desenvolveu um modelo que pode avaliar o a grau de agilidade seguido nos processos de um produto no meio empresarial baseado em quatro dimensões: escopo de método, características da agilidade, valores ágeis e o processo. Este modelo foi avaliado pelo ponto de vista apenas dos gerentes de projeto, faltando assim o uso como prática de GP.

Ganguly (2009) utiliza quatro métricas para avaliar o uso de práticas de gerenciamento de projetos: qualidade do produto; lucratividade; adaptabilidade; e custo. Neste estudo apenas os resultados foram avaliados pelos gerentes de projeto, faltando novamente um exemplo externo.

Binder 2014 desenvolveu um modelo que utilizasse uma combinação das abordagens focando no entendimento da ISO 21500 para grandes organizações com implicações legais e financeiras. O modelo foi desenvolvido sob um estudo de caso, porém não existe referência de uso desse modelo em prática.


\section{CONCLUSÕES}
O uso de abordagens GAP aumentou consideravelmente desde sua oficialização através do Manifesto Ágil. A positividade demonstrada por estudos empíricos no ramo de TI demonstrou que também é possível utilizar estas abordagens para o sucesso de práticas de GP.

Alguns estudos têm apontado a possibilidade de melhorias através do seu uso em projetos de diversas naturezas, outros porém, destacam que continuam utilizando abordagens tradicionais em função de características específicas de projetos de grandes organizações. Assim não é possível determinar se o uso das abordagens GAP vai se espalhar por todos ramos controlando projetos ao longo do globo.

Existem também estudos que estão criando modelos de abordagens que utilizam tanto a base tradicional, quanto técnicas e ferramentas de abordagens GAP. Estes estudos estão utilizando sua própria base de conhecimento para modular seus modelos, entretanto existe vaga menção a aplicação desses modelos na prática através da realização de estudos de caso.

Enfim, através desta pesquisa, realizada meio de levantamento de referências bibliográficas sobre GP, é possível concluir que novos estudos devem ser realizados utilizando como base os novos modelos em projetos de diferentes naturezas.

\singlespacing
\setlength\parindent{0pt}
\section{REFERÊNCIAS BIBLIOGRÁFICAS}

AGUANNO, Kevin. Managing agile projects. Multi-Media Publications Inc., 2005.

ALMEIDA, Luís Fernando Magnanini et al. Fatores críticos da agilidade no gerenciamento e projetos de desenvolvimento de novos produtos. Produto \& Produção, v. 13, n. 1, p. 93-113, 2012.

AMARAL, Daniel Capado et al. Gerenciamento ágil de projetos: aplicação em produtos inovadores. São Paulo: Saraiva, 2011.

ANDERSEN, Erling S. Toward a project management theory for renewal projects. Project Management Journal, v. 37, n. 4, p. 15, 2006.

AUGUSTINE, Sanjiv. Managing agile projects. Prentice Hall PTR, 2005.

BARLOW, Jordan B. et al. Overview and guidance on agile development in large organizations. Communications of the Association for Information Systems, v. 29, n. 2, p. 25-44, 2011.

BATRA, Dinesh et al. Balancing agile and structured development approaches to successfully manage large distributed software projects: A case study from the cruise line industry. Communications of the Association for Information Systems, v. 27, n. 1, p. 21, 2010.

BECK, Kent. Extreme programming explained: embrace change. Addison-Wesley Professional, 2000.

BECK, Kent et al. Manifesto for agile software development. 2001.

BENASSI, João Luís Guilherme et al. EVALUATING METHODS FOR PRODUCT VISION WITH CUSTOMERS’INVOLVEMENT TO SUPPORT AGILE PROJECT MANAGEMENT. In: DS 68-10: Proceedings of the 18th International Conference on Engineering Design (ICED 11), Impacting Society through Engineering Design, Vol. 10: Design Methods and Tools pt. 2, Lyngby/Copenhagen, Denmark, 15.-19.08. 2011. 2011.

BOEHM, Barry W. A spiral model of software development and enhancement.Computer, v. 21, n. 5, p. 61-72, 1988.

BOEHM, Barry. Get ready for agile methods, with care. Computer, v. 35, n. 1, p. 64-69, 2002.

BOEHM, Barry; TURNER, Richard. Using risk to balance agile and plan-driven methods. Computer, n. 6, p. 57-66, 2003.

BOEHM, Barry; TURNER, Richard. Balancing agility and discipline: Evaluating and integrating agile and plan-driven methods. In: Software Engineering, 2004. ICSE 2004. Proceedings. 26th International Conference on. IEEE, 2004. p. 718-719.

BERGGREN, Christian; SÖDERLUND, Jonas. Rethinking project management education: Social twists and knowledge co-production.International Journal of Project Management, v. 26, n. 3, p. 286-296, 2008.

CICMIL, Svetlana JK et al. Exploring the complexity of projects: Implications of complexity theory for project management practice. 2009.

CHARVAT, Jason. Project management methodologies. New Jersey: John Willey \& Sons, 2003.

CHIESA, Vittorio; FRATTINI, Federico. Exploring the differences in performance measurement between research and development: evidence from a multiple case study. R\&D Management, v. 37, n. 4, p. 283-301, 2007.

CHIN, Gary. Agile project management: how to succeed in the face of changing project requirements. AMACOM Div American Mgmt Assn, 2004.

COCKBURN, Alistair. Selecting a project's methodology. IEEE software, n. 4, p. 64-71, 2000.

COCKBURN, Alistair. Crystal clear: a human-powered methodology for small teams. Pearson Education, 2004.

COCKBURN, Alistair. Agile software development: the cooperative game. Pearson Education, 2006.

COHN, Mike. Agile estimating and planning. Pearson Education, 2005.

COLLYER, Simon et al. Aim, fire, aim—Project planning styles in dynamic environments. Project Management Journal, v. 41, n. 4, p. 108-121, 2010.

CONFORTO, Edivandro Carlos; AMARAL, Daniel Capaldo. Evaluating an agile method for planning and controlling innovative projects. Project Management Journal, v. 41, n. 2, p. 73-80, 2010.

CORAM, Michael; BOHNER, Shawn. The impact of agile methods on software project management. In: Engineering of Computer-Based Systems, 2005. ECBS'05. 12th IEEE International Conference and Workshops on the. IEEE, 2005. p. 363-370.

DECARLO, Douglas. Extreme project management: Using leadership, principles, and tools to deliver value in the face of volatility. John Wiley \& Sons, 2010.

DINSMORE, Paul C. et al. AMA-Manual de Gerenciamento de Projetos. Brasport, 2009.

EDER, Samuel et al. Estudo das práticas de gerenciamento de projetos voltadas para desenvolvimento de produtos inovadores. Produto \& Produção, v. 13, n. 1, p. 148-165, 2012.

FERNANDEZ, Daniel J.; FERNANDEZ, John D. Agile project management-Agilism versus traditional approaches. Journal of Computer Information Systems, v. 49, n. 2, p. 10-17, 2008.

FITSILIS, Panos. Comparing PMBOK and Agile Project Management software development processes. In: Advances in Computer and Information Sciences and Engineering. Springer Netherlands, 2008. p. 378-383.

GAREL, Gilles. A history of project management models: From pre-models to the standard models. International Journal of Project Management, v. 31, n. 5, p. 663-669, 2013.

GERALDI, Joana G. et al. Innovation in project management: Voices of researchers. International Journal of Project Management, v. 26, n. 5, p. 586-589, 2008.

HASS, Kathleen B. The blending of traditional and agile project management.PM world today, v. 9, n. 5, p. 1-8, 2007.

HIGHSMITH, Jim; COCKBURN, Alistair. Agile software development: The business of innovation. Computer, v. 34, n. 9, p. 120-127, 2001.

HIGHSMITH, Jim. Agile project management: creating innovative products. Pearson Education, 2009.

STELLINGWERF, Rommert; ZANDHUIS, Anton. ISO 21500 Guidance on project management–A Pocket Guide. Van Haren, 2013.

LINDVALL, Mikael et al. Agile software development in large organizations.Computer, v. 37, n. 12, p. 26-34, 2004.

KOLLTVEIT, Bjørn Johs; KARLSEN, Jan Terje; GRØNHAUG, Kjell. Perspectives on project management. International journal of project management, v. 25, n. 1, p. 3-9, 2007.

LARMAN, Craig; BASILI, Victor R. Iterative and incremental development: A brief history. Computer, n. 6, p. 47-56, 2003.

LEFFINGWELL, Dean. Scaling software agility: best practices for large enterprises. Pearson Education, 2007.

LIN, Ching-Torng; CHIU, Hero; CHU, Po-Young. Agility index in the supply chain. International Journal of Production Economics, v. 100, n. 2, p. 285-299, 2006.

MAGDALENO, Andréa Magalhães; WERNER, Cláudia Maria Lima; DE ARAUJO, Renata Mendes. Reconciling software development models: A quasi-systematic review. Journal of Systems and Software, v. 85, n. 2, p. 351-369, 2012.

PALMER, Steve R.; FELSING, Mac. A practical guide to feature-driven development. Pearson Education, 2001.

POPPENDIECK, Mary. Lean software development. In: Companion to the proceedings of the 29th International Conference on Software Engineering. IEEE Computer Society, 2007. p. 165-166.

PMI, Project Management Institute. Guia PMBOK: um guia do conjunto de conhecimentos do gerenciamento de projetos (5. ed.). São Paulo:  Saraiva, 2014.

SHENHAR, Aaron J.; DVIR, Dov. Reinventing project management: the diamond approach to successful growth and innovation. Harvard Business Review Press, 2007.

SCHWABER, K.; BEEDLE, Mike. gilè Software Development with Scrum. 2002.

SCHWABER, Ken. Agile project management with Scrum. Microsoft Press, 2004.

SLIGER, Michele; BRODERICK, Stacia. The software project manager's bridge to agility. Addison-Wesley Professional, 2008.

SMITH, Preston G. Flexible product development: building agility for changing markets. John Wiley \& Sons, 2007.

SALOMO, Sören; WEISE, Joachim; GEMÜNDEN, Hans Georg. NPD planning activities and innovation performance: the mediating role of process management and the moderating effect of product innovativeness. Journal of product innovation management, v. 24, n. 4, p. 285-302, 2007.

STARE, Aljaž. Agile Project Management in Product Development Projects.Procedia-Social and Behavioral Sciences, v. 119, p. 295-304, 2014.

STYHRE, Alexander. The bureaucratization of the project manager function: The case of the construction industry. International Journal of Project Management, v. 24, n. 3, p. 271-276, 2006.

QUMER, Asif; HENDERSON-SELLERS, Brian. An evaluation of the degree of agility in six agile methods and its applicability for method engineering.Information and software technology, v. 50, n. 4, p. 280-295, 2008.

QUMER, Asif; HENDERSON-SELLERS, Brian. Empirical evaluation of the agile process lifecycle management framework. In: Research Challenges in Information Science (RCIS), 2010 Fourth International Conference on. IEEE, 2010. p. 213-222.

WILLIAMS, Terry. Assessing and moving on from the dominant project management discourse in the light of project overruns. Engineering Management, IEEE Transactions on, v. 52, n. 4, p. 497-508, 2005.

WINTER, Mark et al. Directions for future research in project management: The main findings of a UK government-funded research network.International journal of project management, v. 24, n. 8, p. 638-649, 2006.

WYSOCKI, Robert K. Effective project management: traditional, agile, extreme. John Wiley \& Sons, 2011.

YUSUF, Y. Y. et al. Agile supply chain capabilities: Determinants of competitive objectives. European Journal of Operational Research, v. 159, n. 2, p. 379-392, 2004.
