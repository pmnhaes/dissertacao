\chapter{REVISÃO DA LITERATURA}
\thispagestyle{empty}

\section{Projetos}

\section{Gestão de projetos}

\section{Gestão de Software}

\section{Gestão de Portfolio}

\section{Escritorio de Gestão de Projetos}


\section{Gerenciamento de projetos tradicional}

Ao decorrer dos anos, as abordagens de GP\nomenclature{GP}{Gestão de Projetos} foram disseminadas como “guias de conhecimento”, que apresentavam um conjunto de técnicas, princípios, ações e ferramentas que supostamente seriam capazes de gerenciar qualquer projeto (Kolltveit, 2007; Shenhar, 2007).

Inicialmente essas abordagens mantinham o foco no planejamento e controle do produto de forma engessada sem prever instabilidade e emergências, pois acreditava que era importante entregar o produto de acordo com os requisitos previamente levantados, indiferente a dissociações naturais do dia a dia (Winter, 2006).

Shenhar (2007) afirma que alguns autores investigaram sobre a correlação das práticas adotadas de acordo com o tipo de projeto descrito e das ferramentas utilizadas, e constataram que grande parte dos projetos eram geridos sem a devida diferenciação quanto a sua natureza, e com o uso de práticas genéricas denominadas “melhores práticas”.

De acordo com Collyer (2010) foi reconhecido pelo PMBOK que as práticas deveriam prever planos emergências que lidassem com conformidades não previstas nos requisitos, entretanto quanto a formação da ISO 21500, utilizada como padrão pela GP, nota-se que suas técnicas e ferramentas referiam-se com frequência ao modelo Cascata, quanto ao ciclo de vida do projeto apresentado por Boehm (1988).

Como explica Almeida (2012), é natural que práticas e abordagens evoluam ao longo dos tempos, como projetos passaram a ser o centro de toda e qualquer organização, entretanto, notou-se que a prática de GP se mantinha inadequada e que grande parte das falhas gerências estavam diretamente ligadas as técnicas utilizadas nestas praticas de GP.

A motivação das abordagens de GP consideradas tradicionais estava diretamente relacionada a perspectiva de que projetos eram relativamente simples, previsíveis eque seguiam de forma linear, tornando-os fáceis de detalhar e seguir ao longo do planejamento. Possibilitando assim, uma entrega eficiente dentro do prazo (Andersen, 2006; Boehm, 2002; Boehm \& Turner, 2003; Cicmil, 2009; Collyer, 2010; Leffingwell, 2007; Shenhar, 2007; Williams, 2005; Wysocki, 2011).

Ainda, uma das vantagens descritas na prática tradicional era a capacidade robusta, isto é, a defesa de que diversos projetos poderiam sempre ser gerenciados seguindo as mesmas técnicas e ferramentas, como se fossem todos na mesma natureza.

Para diversos autores (Aguanno, 2004; Cicmil, 2009; Chin 2004; Shenhar, 2007; Williams, 2005; Wysocki, 2011) a robustez de planejamento é a maior desvantagem da prática tradicional, pois projetos têm se tornado cada vez mais complexo, e o para ideal de planejar linearmente não compete lidar com irregularidades dinâmicas da realizada do mercado.

Entretanto, Styhre (2006) destaca que no desejo por diminuir riscos e evitar ineficiências, o foco de gerenciar criatividade e mudanças é burocratizado e acaba por ser tornar um gerenciamento de papéis e formulários, geralmente encontrados na robustez do projeto.

Outra desvantagem notada é que por focar no planejamento e controle, essas abordagens ignoram não só a natureza do projeto, como também aspectos humanos quanto a sociabilização, contextualização e tendência a mudanças (Winter, 2006; Highsmith, 2009).


\section{Gerenciamento de projetos ágil}
Em função as desvantagens apontadas nas abordagens tradicionais de GP, somadas ao crescimento de um mercado inovador, com demandas contínuas e necessidade pela redução de custos, resultaram no advento de novas abordagens para GP (Aguanno, 2004; Conforto \& Amaral, 2010; Williams, 2005).

Em 2001, a partir de uma iniciativa em conjunto conhecida por Manifesto Ágil houve a formalização das abordagens ágeis. Seu diferencial se apresentava, simplificadamente, como um conjunto de técnicas, princípios e ferramentas que melhoravam o trabalho em time, permitindo adaptação e evoluções ao longo do projeto (Beck et al., 2001; Berggren, 2008; Cohn, 2005; Hass, 2007; Highsmith, 2009; Fernandez, 2008; Fitsilis, 2008; Larman, 2003; Salomo, 2007; Schwaber, 2004; Smith, 2007; Qumer, 2010).

De acordo com Cockburn (2000) e Schwaber (2001) existem quatro princípios ágeis que devem sempre ser enfatizados:

\begin{enumerate}
    \item{Indivíduos e Interações são mais importantes que processos e ferramentas;}
    \item{Colaboração com as necessidades dos clientes acima de negociação de contratos;}
    \item{Produto em funcionamento vale mais que documentação abrangente;}
    \item{Responder as mudanças independente do plano a seguir.}
\end{enumerate}

Existem também práticas e ferramentas, como a concepção de visão do produto, desenvolvimento iterativo; além do uso de ferramentas visuais de planejamento do projeto e de seus artefatos (Augustine, 2005; Boehm \& Turner, 2004; Chin, 2004; Highsmith, 2009).

Ao ressaltar a importância dos itens, entretanto, os autores destacam que eles não se excluem, o ideal é que todos sejam levados em consideração, apenas seguindo a ordem de importância (Aguanno, 2004).

Essas abordagens podem ser apontadas por GAP, e representaram uma alternativa para problemática das mudanças constantes, alterações a curto prazo dentro do planejamento do projeto, valorização do cliente como uma fonte de interação (Amaral, 2011; Augustine, 2005; Cohn, 2005; Highsmith, 2009).

Alguns autores (Aguanno, 2004; Boehm, 1988; Manifesto, 2001; Williams, 2005), enfatizam que estas abordagens têm uma relação muito estreita com a engenharia de software, levando essas abordagens a aparecerem com frequência relacionadas ao desenvolvimento de software.

Stare (2014) afirma que durante uma entrevista os autores responsáveis pelo Manifesto Ágil ressaltaram a importância das abordagens GAP como um meio de tornar o GP uma ferramenta de sucesso no atual mercado, e que apesar de serem inicialmente aplicadas em projetos de tecnologia de informação (TI), eram igualmente apropriadas a qualquer natureza de projetos, visto que o foco destas abordagens está em reconhecer e aplicar feedback com a finalidade de lidar com incertezas.

Apesar desta declaração, Stare (2014) contesta que nos estudos pesquisados até o ano de 2009, em sua maioria, as práticas ágeis se referiam ao uso em projetos de TI, e que neste ano, algumas pesquisas começaram a reconhecê-las em outros campos com certo receio.

Entre as metodologias ágeis mais utilizadas, as que mais se destacam na indústria são: Scrum (Schwaber, 2004), Lean Software Development (Poppendieck, 2003), Crystal (Cockburn, 2004), Feature Driven Development (Palmer, 2002), Adaptive Software Development (Highsmith, 2001) e eXtreme Programming (Beck, 2000).

Considerando a crescente necessidade da indústria por inovação e pela necessidade de redução de custos, as abordagens GAP tomaram o cenário de GP devido sua habilidade de adaptação durante o ciclo de vida dos projetos, indiferente a sua natureza. Mudanças são inevitáveis e impossíveis de prever, portanto é importante lidar com elas (Aguanno, 2004; Conforto \& Amaral, 2010; Chin, 2004; DeCarlo, 2004; Highsmith, 2009; Leffingwell, 2007; Williams, 2005).

Além disso, como atrativos para utilização de práticas ágeis se destacam: a melhora na habilidade de comunicação formal e informal, bem como a colaboração externa com a aproximação dos clientes no processo de produtos (Aguanno, 2004; Cockburn, 2006; Collyer., 2010; Coram, 2005; DeCarlo, 2004; Highsmith, 2001; Williams, 2005).

Alguns autores destacam que a proximidade do time com o projeto melhora seu desempenho e a habilidade de ser autogerenciáveis e responsáveis em suas tarefas (Augustine, 2005; Boehm \& Turner, 2004; Highsmith, 2009).

Vale destacar que também existiram ideias de foram herdadas das práticas tradicionais, como o ideal de que o projeto fosse iterativo, isto é, pudesse ser incrementado ao longo do processo de produção (Boehm, 1988; Aguanno, 2004).

Entretanto, invés de utilizar um único plano para o projeto, GAP utilizam a ideia de iteratividade ao longo de todo processo através de pequenas fases com retorno geralmente fornecido por feedback (Boehm \& Turner, 2004; Highsmith, 2009; Schwaber, 2004; Augustine, 2005; Cohn, 2005).

Para Poppendieck (2003) o motivo pelo qual as práticas ágeis se destacaram esta diretamente relacionado a sua capacidade de entregar produtos com rapidez e qualidade, atendendo as necessidades do cliente de forma satisfatória utilizando princípios que já haviam sido citados pela metodologia Lean de produção.

 A marca de agilidade é também propiciar uma melhoria de performance a medida que o cliente colabora mais ativamente frente a demonstração do produto; além de trazer o conceito de flexibilidade e estabilidade. (Chin, 2004; Dorairaj, 2012).

Para Aguanno (2004) uma das maiores vantagens do da GAP é a redução de riscos e falhas, a partir do momento que seu escopo pode ser definido ao longo do projeto, mudanças não se tornam transtornos e portanto não são erros.

Entre outras vantagens, é importante destacar o uso de ferramentas visuais que permitem organizar ideias, deixar claros objetivos e fases, tornar processos compreensíveis e facilitar o planejamento de projetos e gestão de portfólios, reduzindo riscos burocráticos (Malachia, 2013).

Alguns autores discutem que é preciso que as mudanças não sejam aplicadas apenas pela inovação do mercado, mas também na forma de pensar em GP, e consequentemente na estrutura da própria organização para que haja o correto emprego dos princípios e ferramentas observados (Aguanno, 2004; Boehm \& Turner, 2003; Chin, 2004; Cockburn, 2006; DeCarlo, 2004; Highsmith, 2009; Lawrence, 2006; Leffingwell, 2007; Shenhar, 2007).

Finalmente é preciso questionar se as práticas e princípios previstos nas abordagens GAP realmente se aplicam ao mercado, e se ao aplicá-los será possível entregar produtos de qualidade, com melhor custo e dentro do prazo.

\section{Repensando a Gestão de Projetos}

Através de estudos, alguns autores (Coram, 2005; Conforto \& Amaral, 2010; Leybourne, 2009) notaram que alguns projetos procuram utilizar GAP como uma forma de fugir de ferramentas e técnicas particulares empregadas pelas abordagens tradicionais, e acreditam que estudos mais detalhados deveriam ser realizados nesses termos.

Para Cockburn (2000) toda abordagem tem seus limites, por mais bem embasada que possa ser, principalmente quando se tratam de produtos e clientes. Para que um determinado produto seja entregue com sucesso, é reconhecido que é ideal o uso de uma abordagem de GP, entretanto tanto a abordagem tradicional quanto a abordagem ágil possuem vantagens e desvantagens (Aguanno, 2004; Andersen, 2006).

Devido o aumento da demanda de inovação o crescimento do uso das abordagens, tanto individualmente quanto mescladas, tem sido o foco de estudos empíricos que buscam adaptá-las a diferentes tipos de projetos (Conforto \& Amaral, 2010).

De acordo com o estudo de Benassi (2011) existe evidencia de aspectos positivos no que diz respeito ao uso de GAP, entre elas: aumento na velocidade em que empresas alcançam inovações; cortes em custos excessivos, entre eles custos de armazenagem e desenvolvimento; diminuição do tempo de entrega do produto; reduz falhas no atendimento aos requisitos e portanto atende melhor ao desejo do cliente ao entregar produtos de qualidade em um prazo pequeno.

Entretanto, foi identificado também que a sugestão de pouca documentação em projetos é sugerida apenas para organização que possuem times pequenos e/ou bem organizados, que não detenham restrições corporativas e de procedimentos (Lindvall, 2004; Boehm \& Turner, 2003).

Alguns estudos definiram que projetos cujo o escopo é bem definido inicialmente, onde requisitos e metas possuem baixa nível de incerteza e mudança; e que podem ser desenvolvidos sem a presença frequente do cliente podem ser bem sucedidos com o uso de abordagens tradicionais, pois precisam focar apenas no planejamento e na otimização de atividades previstas (Boehm, 2002; Coram, 2005; DeCarlo, 2004; Fernandez, 2008; Shenhar, 2007; Wysocki, 2011).

Consequentemente, se o projeto for realizado em grandes corporações, independente de seu tamanho em particular, seu grau de complexidade ou sua duração; não é aconselhável que sejam utilizadas apenas abordagens GAP (Aguanno, 2004; Boehm, 2002; Boehm \& Turner, 2003; Cockburn, 2000; Fowler, 2005; Highsmith, 2009).

Para flexibilizar produtividade e satisfazer as necessidades de grandes organização, é sugerido que modelos sejam desenvolvidos, mesclando um pouco das ferramentas, técnicas e princípios de abordagens tradicionais e com GAP (Boehm \& Turner, 2003; Batra, 2010; Conforto \& Amaral, 2010; Barlow, 2011; Magdaleno, 2011).

Estudos recentes vêm utilizando referências teóricas e estudos empíricos para desenvolver modelos que sigam um pouco de cada abordagem (Lindvall, 2004; Batra , 2010; Barlow , 2011).

Mafakheri (2008) avalia o grau de agilidade de projeto, sob seis características: dinamismo; tamanho da equipe; comunicação; capacidade de testar resultados; conhecimento e habilidades relacionados ao produto. Não foi apresentado um estudo de caso comprovando a eficiência do modelo como uma prática de GP.

Qumer \& Henderson-Sellers (2008) desenvolveu um modelo que pode avaliar o a grau de agilidade seguido nos processos de um produto no meio empresarial baseado em quatro dimensões: escopo de método, características da agilidade, valores ágeis e o processo. Este modelo foi avaliado pelo ponto de vista apenas dos gerentes de projeto, faltando assim o uso como prática de GP.

Ganguly (2009) utiliza quatro métricas para avaliar o uso de práticas de gerenciamento de projetos: qualidade do produto; lucratividade; adaptabilidade; e custo. Neste estudo apenas os resultados foram avaliados pelos gerentes de projeto, faltando novamente um exemplo externo.

Binder 2014 desenvolveu um modelo que utilizasse uma combinação das abordagens focando no entendimento da ISO 21500 para grandes organizações com implicações legais e financeiras. O modelo foi desenvolvido sob um estudo de caso, porém não existe referência de uso desse modelo em prática.

