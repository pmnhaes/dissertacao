\chapter{ESTUDO DE CASO}
\thispagestyle{empty}

\section{A natureza de pesquisa}

Esta pesquisa pode ser caracterizada como uma pesquisa bibliográfica, cujo método utilizado foi o estudo de caso. Visando atingir o objetivo da pesquisa, inicialmente, foi realizada uma pesquisa bibliográfica através do material existente na literatura.

Através da leitura foi possivel identificar os elementos de pesquisa propostos por \citeonline{lakatos2010fundamentos}, são eles: intenção, reflexão espírito crítico, atenção, análise e síntese.

Se utilizada a proposta de \citeonline{gil2002elaborar}, a presente pesquisa é tomada por qualitativa quanto a abordagem da problemática, e descritiva quanto a caracterização de seus objetivos, pois utiliza do desenvolvimento de um estudo de caso.

\citeonline[p 239]{fortin2009fundamentos} confirma que todo estudo que objetiva a identificação de características referentes a um fenómeno, visando obter uma visão geral de sua situação ou população em questão, pode ser apresentado como um estudo descritivo. Nestas condições, determinado trabalho consistirá na pretenção de descrever ou interpretar a propriedades desta investigação, através de análises empíricas e da descrição da problemática \cite{fortin2009fundamentos, lakatos2010fundamentos}.

\section{Método de Pesquisa}

Para \citeonline{de2007metodologia} todo e qualquer trabalho científico deve proporcionar a reprodução de experiências de modo que outros pesquisadores sejam capazes de obter resultados descritivos, de repetir suas observações e ainda de realizar julgamentos as conclusões de seu(s) autor(es).

De acordo com \citeonline{vergara2009projetos} pesquisas também podem ser caracterizadas em relação aos aspectos relativos aos fins e aos meios. A presente pesquisa influe no carater exploratório descritivo quanto aos fins, pois se enquadra expõem características de uma determinada população, sem o compromisso de explicar os fenômenos que descreve, embora sirva de base para essa explicação.

Assim, porque é de pretensão deste trabalho descrever e interpretar, mas do que avaliar um caso real presente, e seguindo o caráter de investigação revelado por \citeonline{fortin2009fundamentos}, este trabalhor se apresenta como descritivo e exploratório.

Por sua vez, quanto aos procedimentos técnicos que foram utilizados para obtenção dos dados, seguindo a classificação proposta por \citeonline{lakatos2010fundamentos}, trata-se de pesquisa bibliográfica, que também pretende utilizar de uma pesquisa de campo.Uma pesquisa de campo é aquela que busca alcançar informações referentes a um problema e busca uma resposta ou de uma hipótese, que se pretende comprovar, bem como descobrir novos fenômenos ou as relações entre eles \cite{de2007metodologia}.


\section{População e Amonstragem}

\citeonline[p 20]{de2011elaboraccao} define população pesquisa como aqueles a que ela se refera ou representa, isto é, o universo compreendido dentro da pesquisa. Nesta pesquisa, o universo abordado se refere aos funcionários envolvidos no processo de construção do escritório de projetos do PICG.

Para atender a discussão científica proposta no presente estudo, de acordo com a posição de \citeonline{vergara2009projetos}, foi apresentada uma problemática referente ao universo que será exposta no próximo capítulo.
