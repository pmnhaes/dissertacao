\chapter{ESTUDO DE CASO}
\thispagestyle{empty}

\section{Objeto de estudo}

Para Ristoff (2011) a Reforma Universitária garante que as universidades federais terão, enfim, a autonomia de gestão financeira prevista na constituição, mas nunca posta em prática; terão asseguradas a tão sonhada dotação global de recursos, a irredutibilidade nos repasses e a expansibilidade continuada.

Estarão, portanto, livres das amarras burocráticas e financeiras, que inibem a autogestão, repelem a inovação e forçam a privatização do espaço público. Ao defender a autonomia das universidades, nos termos do artigo 207 da Constituição, o governo também deixa claro o seu entendimento de que nestas instituições, embora não necessariamente nos Centros Universitários e Faculdades, as atividades de ensino, pesquisa e extensão são definidoras de sua natureza.

Em 2008, foram fundamentados os Institutos Federais (IF) no país, para que fosse possível dar mais um passo no processo de expansão da Rede Federal de Educação, e também como parte dos objetivos de Plano de Desenvolvimento da Educação (PDE) (BRASIL, 2016).

Um dos incentivos governamentais para conciliação de universidade, ou institutos, com empresas veio através da formalização da Empresa Brasileira de Pesquisa e Inovação Industrial (EMBRAPII) em 2013, para fomentar o processo de cooperação entre pequenas  e medias empresas nacionais e instituições tecnológicas ou privadas sem fins lucrativos.

Os primeiros projetos pilotos envolvem o Instituto de Pesquisa Tecnológico (IPT) na área de nanobiotecnologia, o Instituto Nacional de Tecnologia (INT) em energia (gás/petróleo) e saúde e o Centro Integrado de Manufatura e Tecnologia (CIMATEC) do Serviço Nacional de Aprendizagem Industrial (SENAI) na área de automação manufatura (EMBRAPII1, 2016).

Em março de 2015, a EMBRAPII selecionou cinco IF para atuarem em projetos de inovação industrial, para passarem pelo processo de seleção foi exigido das IF que participassem de um curso de capacitação de seus agentes de inovação, para que estivessem preparados para interagir com a relação UEG visando proporcionar eficiência  e agilidade no processo de transmissão de tecnologia para sociedade (EMBRAPII1, 2016).

Entre os polos selecionados destacasse o Polo de Inovação Campos dos Goytacazes pertencente ao Instituto Federal Fluminense (IFF) em parceria com o Campus Rio Paraíba do Sul (UPEA), inicialmente Unidade de Pesquisa e Extensão Agroambiental.

O polo está  localizado no município de Campos dos Goytacazes – RJ, no norte do estado do Rio de Janeiro, e foi reconhecido pelo Ministério da Educação em 13 de agosto de 2015 (PICG/IFFluminense – Portaria 819/2015). Desde sua inauguração a UPEA vem  realizando diversos trabalhos de fundamento ambiental para o atendimento das demandas regionais (EMBRAPII2, 2016).

Neste momento o PICG está sendo estruturado para desenvolver projeto de PD\&I e receberá um financiamento de R\$ 3 milhões para um plano de ação de 3 anos. Este plano de ação contará diretamente com a participação de empresas da região, visando transferir tecnologia para essas e, por fim, para a sociedade local (IFF, 2016).

