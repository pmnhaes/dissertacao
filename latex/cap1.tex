\chapter{INTRODUÇÃO}
\thispagestyle{empty}

\section{Contextualização}
Com a chegada do século 21, o mundo enfrentou diversas mudanças, tanto com respeito as inovações tecnológicas quanto ao cenário no mercado produtivo. Custo e qualidade deixaram de ser a principal motivação de melhoria nas empresas, cedendo espaço para projetos que entreguem produtos mais flexíveis em menor espaço de tempo \cite{yusuf2004agile}.

Neste contexto foi possível destacar a importância da inovação, do atendimento as constantes modificações de requisitos por parte dos clientes e ainda importância de manter o cliente como alvo da organização, para conquistá-lo e gerar vínculos de fidelização \cite{chiesa2007exploring}.

O aumento da concorrência dos produtos nacionais com os equivalentes importados, também obriga as empresas a buscarem inovação na distinção de seus produtos frente a competitividade de seus concorrentes \cite{atkinson2012global}.

Esse aumento do uso da inovação como principal fator de crescimento econômico traz a necessidade de uma forma de controle as regras de crescimento e a coordenação dos Sistemas de Inovação (SI)\nomenclature{SI}{Sistemas de Inovação} que tendem a surgir para que promovam a sustentabilidade e competitividade a longo prazo \cite{lundvall2010politicas}.

No Brasil, com a abertura do país ao comércio internacional, ocorre também o aumento dos incentivos governamentais para a pesquisa e desenvolvimento e a procura por inovação através de parcerias com universidades por parte das empresas resultando numa relação de aproximação. Cada vez mais as empresas, universidades e governos intensificam as suas interações através dos SI.

A empresa busca a inovação que permitirá a sua sobrevivência no mercado e o aumento de suas margens de lucros; a universidade encontra neste ambiente condições para a obtenção de mais financiamento para a pesquisa e também recursos para o financiamento, sua participação no sistema também acarreta uma maior aproximação com a sociedade na medida que os resultados da pesquisa tenham um aproveitamento mais direto pela mesma; enquanto isso, universidade e governos interagem em busca de resultados alinhados com os objetivos de cada um e, por derradeiro, os governos, que nas suas diversas esferas buscam o desenvolvimento local, regional e nacional e da sociedade como um todo.

Desde o final da década de 1990, as politicas de Ciência, Tecnologia e Inovação (CT\&I)\nomenclature{CT\&I}{Ciência, Tecnologia e Inovação} vinham sendo implementadas no país seguindo a lógica de modelo linear de inovação. Juntamente, as politicas de Pesquisa e Desenvolvimento (P\&D)\nomenclature{P\&D}{Pesquisa e Desenvolvimento} que visavam articular a relação entre universidades, empresas e governo através dos recursos previstos no Sistema Nacional de Inovação (SNI)\nomenclature{SNI}{Sistema Nacional de Inovação} \cite{kuhlmann112008logicas}.

Com o tempo, essas interações passaram a ser reconhecidas por interações de Pesquisa, Desenvolvimento e Inovação (PD\&I)\nomenclature{P\&DI}{Pesquisa, Desenvolvimento e Inovação} e se responsabilizaram por gerar a produção de novos conhecimentos, proporcionando a transferência de tecnologia do meio acadêmico para as empresas e também a proteção destes ativos intelectuais através dos direitos de propriedade intellectual, mediante a obtenção de patentes e registro de direitos autorais. Ao incorporar o termo inovação, é possível destacar três aspectos fundamentais: interação com a sociedade; com as empresas; e com o governo. Em outras palavras, inovação significa P\&D mais transferência de tecnologia.

Os riscos mais comuns ao sucesso da relação Universidade-Empresa-Governo (UEG)\nomenclature{UEG}{Universidade-Empresa-Governo} estão ligados as universidades que precisam se adequar para corresponder a demanda da sociedade; criar mecanismos de pesquisa e desenvolvimento interdisciplinares com múltiplas fontes de fomento, sejam governo, empresas ou instituições; estimular pesquisas prioritárias que aloquem necessidades de forma planejada; criar mecanismos de proteção a propriedade intellectual; lidar com a transferência das tecnologias criadas; e ainda estimular acadêmicos pelo caminho empreendedor a fim de gerar oportunidades.

Ao passo que a relação UEG passou a representar uma solução para a necessidade de inovação de produto no mercado, também houve uma crescente busca por soluções que tornassem possível gerenciar esta inovação em ambientes desafiadores e imprevisíveis, advindo dos processos da mesma, a indústria decidiu voltar-se ao GP\nomenclature{GP}{Gestão de Projetos}, que no entanto, de acordo com \citeonline{geraldi2008innovation}, mostrou-se incapaz de lidar com tais mudanças a princípio.

Para \citeonline{yusuf2004agile} as ferramentas e técnicas tradicionais empregadas no GP não previam atividades de inovação nem instabilidades durante o processo de seus produtos, pois previam projetos linear e sem inconstâncias.

Visando atender a expectativa de mercado e a melhoraria da gestão de projetos, abordagens alternativas, que vinham se desenvolvendo desde o início dos anos 90, com novos princípios, ferramentas e técnicas, pareceram a melhor resposta. Essas abordagens vieram a ser conhecidas por Gestão Ágil de Projetos (GAP)\nomenclature{GAP}{Gestão Ágil de Projetos} \cite{amaral2011gerenciamento}.

Entretanto é preciso que haja melhor compreensão sobre cada abordagem no que respeito aos processos executados na formação dos produtos em seu meio, e verificada adaptação destas abordagens em meio a inovação.

\section{Problemática}

No mercado moderno a competitividade através da inovação está diretamente baseada no conhecimento e portanto na ciência e a tecnologia que são fatores básicos para a geração desta. A relação UEG, conhecida por tríplice hélice, se mostrou essencial para a transmissão do conhecimento de volta para o mercado e portanto para a sociedade.

Assim, cada vez mais as parcerias entre instituições de ensino direcionadas à pesquisa científica e tecnológica e empresas privadas têm ganhado destaque, visto a possibilidade de investimento de recursos para o desenvolvimento de soluções inovadoras para a sociedade.

Para garantir a viabilidade destas parceria e expor as vantagens obtidas nesta relação, diversas ferramentas devem ser postas em ação, entre essas ferramentas encontra-se o uso de abordagens de gerenciamento de projetos.

O uso de abordagens GAP aumentou consideravelmente desde sua oficialização através do Manifesto Ágil. A positividade demonstrada por estudos empíricos demonstrou que é possível utilizar estas abordagens para o sucesso de práticas de GP.

Alguns estudos têm apontado a possibilidade de melhorias através do seu uso em projetos de diversas naturezas, outros porém, destacam que continuam utilizando abordagens tradicionais em função de características específicas de projetos de grandes organizações. Assim não é possível determinar se o uso das abordagens GAP vai se espalhar por todos ramos controlando projetos ao longo do globo.

Existem também estudos que estão criando modelos de abordagens que utilizam tanto a base tradicional, quanto técnicas e ferramentas de abordagens GAP. Estes estudos estão utilizando sua própria base de conhecimento para modular seus modelos, entretanto existe vaga menção a aplicação desses modelos na prática através da realização de estudos de caso.

\section{Objetivo Geral}

Verificar através de um mapeamento e de um estudo de caso se os conceitos e ferramentas empregados no gerenciamento de projetos dentro de um PMO,a valia de uma relação UEG, podem realmente contribuir na busca por inovação e competitividade do mercado de trabalho.


\section{Objetivos específicos}

\begin{itemize}
  \item{Avaliar o uso de abordagens de GP no PMO;}
  \item{Avaliar a filosófica das abordagens de GAP;}
  \item{Avaliar a diferença de produtividade implícita no PMO;}
  \item{Levantar literatura científica sobre o tema;}
  \item{Observar em campo a aplicação das abordagens de GP num PMO, analisar e discutir seus resultados;}
  \item{Verificar a adequação no uso de práticas de GAP;}
  \item{Verificar a satisfação das empresas envolvidas na relação UEG;}
  \item{Verificar a adequação do Polo de Inovação Campos dos Goytacazes (PICG)\nomenclature{PICG}{Polo de Inovação Campos dos Goytacazes} frente ao plano de ação do governo brasileiro;}
  \item{Analisar o impacto regional frente a criação do PICG.}
\end{itemize}

\section{Justificativa}

Este trabalho apresenta uma visão sobre abordagens de gestão de projetos, apontando vantagens e desvantagens de seu uso, e ainda sobre trabalhos que utilizaram técnicas de ambas abordagens, propondo novos modelos de seu uso. É possível argumentar que muitos projetos ainda não utilizam nenhuma abordagem, e que para os que utilizam, a mesclagem delas é aconselhável.

Recentemente a região de Campos do Goytacazes recebeu a abertura de um Polo de Inovação (PI)\nomenclature{PI}{Polo de Inovação} para facilitar na relação UEG. Observa-se um ambiente propício para esse tipo de interação a partir de algumas iniciativas que vem sendo promovidas pelo PICG, sendo notado também a necessidade de estudos aprofundados através de trabalhos futuros para análise do impacto da criação do PI.

A contribuição almejada neste trabalho será a modelagem de uma abordagem para o escritório de projetos de GP do PICG, frente a uma abordagem mesclada de práticas de gerenciamento de projeto tradicionais e ageis, que visam possibilitar a entrega da inovação dentro de uma relação UEG.


\section{Estrutura do trabalho}

Este projeto está organizado em 5 capítulos. Neste primeiro capítulo foi apresentado o contexto no qual a pesquisa se encaixa frente as motivações do trabalho e seus objetivos.

No segundo capítulo é apresentada uma revisão bibliográfica no que respeito as práticas de GP, relacionando sua criação e conceitos.

O terceiro capítulo dá continuidade a revisão bibliográfica do projeto, porém relatando sobre a relação UEG, na sua criação e presença no país, além da importância para o universo de estudo.

No quarto capítulo demonstra-se sobre a metodologia aplicada no estudo, frente ao objeto do estudo de caso, o PICG.

O ultimo capítulo representa um espaço para discussão dos resultados a serem obtidos nesta pesquisa e portanto as considerações finais.

