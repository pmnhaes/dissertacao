\chapter{INTRODUÇÃO}
\thispagestyle{empty}

\section{Contextualização}
Quando custo e qualidade deixaram de ser a principal motivação de melhoria nas organização, e a importância dos processos inovação como meio de distinção de produto e manutenção da competitividade foi salientada junto a necessidade do atendimento às constantes modificações de requisitos por parte dos clientes, que se destacaram ainda mais como alvo das organizações \cite{atkinson2012global}.

Para os autores \citeonline{prado2006mmgp, pmiguide2014}, apenas por meio da realização de projetos que empresas se tornam capazes de realizar as mudanças necessárias e atender às crescentes exigências dos consumidores por obter produtos, ou serviços, de excelência, dentro do prazo e custo estabelecidos adequadamente.

É possível notar que nos últimos anos, tanto em organizações privadas quanto públicas, existe uma necessidade de estabelecer melhores ações estratégicas que sejam capazes de responder de forma àgil e eficaz as mudanças propostas nos projetos, porém que não concluam na insatifação dos resultados desses projetos \cite{de2008construindo}.

Frente a problemática da satisfação do cliente com as entregas efetuadas pelos projetos, e procurando viabilizar ferramentas, técnicas e soluções para a gestão de projetos (GP), alguns autores buscaram apresentar processos que representam melhores que práticas e que por vezes incluem modelos de maturidade como meio de vincular a melhoria na performance dos projetos a evolução de uma escala de aderência às práticas recomendadas \cite{kerzner2013project, pmiguide2014, prado2006mmgp}.

Entretanto, \citeonline{atkinson2012global} demonstraram uma certa preocupação de que essas abordagens se envolvessem mais com a eficiência do projeto, do que com o atendimento às expectativas dos participantes envolvidos na criação deles, entre os quais o cliente mais uma vez se destaca.

A estrutura de gerenciamento chamada de escritório de gestão de projetos (EGP) passou a se destacar por representar uma padronização dos processos de governança relacionados ao projetos que visam o compartilhamento de recursos, metodologias, ferramentas e técnicas. Assim, essa estrutura pode ser responsável por um ou mais projetos, auxiliando-os a alcançar seus objetivos com excelência, de acordo com seus requisitos e aumentando a taxa de sucesso \cite{pmiguide2014}.

Em outro ponto, \citeonline{garnica2009gestao} demonstrou preocupação quanto a gestão de projetos em nível tecnólogico no contexto das instituições de ensino, que vêm se destacando como meio de conectar atividades de pesquisa a utilização de novos conhecimentos para a sociedade, provendo assim fontes de inovação.

Esse aumento da dependência da inovação, advinda do meio da pesquisa, como principal fator de crescimento econômico traz a necessidade de uma forma de controle as regras de crescimento e a coordenação dos Sistemas de Inovação (SI) que tendem a surgir para promover a sustentabilidade e competitividade a longo prazo \cite{lundvall2010politicas}.

No Brasil, a Associação Brasileira de Pesquisa e Inovação Industrial (EMBRAPII), se destaca por promover a união igualitária de responsabilidade de financiamento de dois importantes orgãos federais, o Ministério da Ciência, Tecnologias, Inovações e Comunicações (MCTIC) e o Ministério da Educação (MEC). Além deste envolvimento, em recente comunicado a associação anunciou a parceria com o Instituto Federal Fluminense (IFF), para formação do Polo de Inovação Campos dos Goytacazes (PICG), e salientou sua missão de contribuir para o desenvolvimento da inovação na industria através da pesquisa.

Para averiguar a efetividade dos processos de gestão de projetos, dentro de um escritório de gestão de projetos do Polo de Inovação, este trabalho pretende apresentar uma revisão da literatura no quesito da gestão de projetos, e o contexto para a implementação do mapeamento da maturidade desses projetos frente ao modelo EMBRAPII.

% \section{Problemática}

% o modelo EMBRAPII foi desenvolvido levando em consideração experiência passada mas se conferir se, ou sem ser comparado com um modelo conhecido e abrangido internacionalmente, como o CMMI

\section{Objetivo Geral}
% O objetivo deste trabalho é avaliar se a evolução na maturidade em gerenciamento de projetos conduz a melhorias no gerenciamento e, consequentemente, no nível de desempenho dos projetos, bem como identificar as principais dificuldades encontradas neste processo de evolução

Verificar através de um mapeamento e de um estudo de caso se os conceitos e ferramentas empregados na gestão de projetos dentro de um escritório de projetos, podem efetivamente contribuir na busca por inovação e competitividade do mercado.


\section{Objetivos específicos}

\begin{itemize}
  \item{Levantar literatura científica sobre o tema;}
  \item{Observar em campo a aplicação de abordagens de gestão de projetos em um escritório de projetos;}
  \item{Analisar e discutir sobre o uso dessas abordagens;}
  \item{Avaliar a diferença de produtividade implícita no escritório de projetos;}
  \item{Verificar a adequação da mesclagem de práticas tradicionais e agéis da gestão de projetos;}
  \item{Verificar a adequação do Polo de Inovação Campos dos Goytacazes (PICG) frente ao plano de ação da EMBRAPII.}
\end{itemize}


\section{Justificativa}

Este trabalho apresenta uma visão sobre abordagens de gestão de projetos, referenciando também a gestão de programas e portfólio, bem como seu enquadramento nos escritórios de gestão de projetos. Apresenta-se também o objeto de caso de estudo, o PICG, um polo de inovação com recente implantação de um escritório de projetos, que se aproxima de sua primeira entrega.

A contribuição almejada neste trabalho será o mapeamento de um modelo de maturidade de projetos, imposto pela EMBRAPII, para contribuir com a representação da efetividade da GP neste ambiente.

\section{Estrutura do trabalho}

Este projeto está organizado em quatro capítulos, um apendíce e um anexo. No primeiro capítulo foi apresentado o contexto no qual a pesquisa se encaixa frente as motivações do trabalho e seus objetivos.

O segundo capítulo apresentada uma revisão bibliográfica no que respeito as práticas de gestão de projeto, relacionando seus conceitos, motivações e situação atual.

O terceiro capítulo propõe a metodologia que será empregada no trabalho, destacando sua natureza, sua população e amostragem, frente ao objeto do estudo de caso, o PICG.

O ultimo capítulo representa um espaço para discussão dos resultados a serem obtidos nesta pesquisa, com as considerações finais e um cronograma de execução do projeto.

No apendíce encontra-se uma análise bibliográfica sobre o tema, e inclui uma apresentação dos periódicos escolhidos para esse projeto.

O anexo representa o artigo desenvolvido, a partir de uma revisão bibliográfica de temática às práticas de gestão de projetos ao longo dos anos, para cumprimento das exigências a apresentação deste projeto.
