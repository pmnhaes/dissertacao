\chapter{CONSIDERAÇÕES FINAIS}
\thispagestyle{empty}

Ao consultar o Portal de Inovação e Diretório de dados do CNPq, é possível notar que o Brasil possui inúmeros pesquisadores em diversas áreas do conhecimento, bem como diversos projetos de pesquisa, incubadoras e parques tecnológicos dignos de primeiro mundo.

Este trabalho apresentou uma leve perspectiva sobre um novo plano de ação do governo brasileiro envolvendo as novas IF, reformuladas e focadas em ensino de qualidade, e de Polos de Inovação, idealizados para reger as relações estabelecidas entre os institutos e as empresas locais onde estão localizados.

Para atender as expectativas de inovação e competitividade impostas ao PICG, este trabalho irá buscar verificar o uso de conceito e ferramentas de gestão de projetos empregados no escritório de projetos, através de um mapeamento com base no modelo de maturidade imposto pela EMBRAPII.


\newpage
\thispagestyle{empty}
\singlespacing
\section{Cronograma}

\begin{enumerate}
  \item{Contato – Equipe Administrativa do Polo de Inovação Campos dos Goytacazes (PICG);}
  \item{Participação treinamento de GP do PICG;}
  \item{Revisão Bibliográfica;}
  \item{Coleta de dados – escritório de projetos;}
  \item{Roteirização;}
  \item{Defesa do Projeto;}
  \item{Análise dos Indicadores dos projetos do PMO;}
  \item{Discussão sobre análise;}
  \item{Discussão dos Resultados;}
  \item{Revisão do projeto final;}
  \item{Defesa.}
\end{enumerate}

\begin{table}[!htpb]
  \centering
  \begin{small}
    \setlength{\tabcolsep}{4pt}
    \begin{tabular}{|c|c|c|c|c|c|c|c|c|c|c|c|c|c|c|c|c|c|}\hline
      & \multicolumn{17}{c|}{Meses (2015 - 2016)}\\ \cline{2-18}
      \raisebox{1.5ex}{Etapa} & ago & set & out & nov & dez & jan & fev & mar & abr & mai & jun & jul & ago & set & out & nov & dez \\ \hline
      1 & X &   &   &   &   &   &   &   &   &   &   &   &   &   &   &   &   \\ \hline
      2 & X & X & X & X & X & X &   &   &   &   &   &   &   &   &   &   &   \\ \hline
      3 &   &   & X & X & X & X & X & X & X & X & X & X &   &   &   &   &   \\ \hline
      4 &   &   &   &   &   &   &   &   & X & X &   &   &   &   &   &   &   \\ \hline
      5 &   &   &   &   &   &   &   &   &   & X & X &   &   &   &   &   &   \\ \hline
      6 &   &   &   &   &   &   &   &   &   &   &   &   & X &   &   &   &   \\ \hline
      7 &   &   &   &   &   &   &   &   &   &   &   & X & X & X &   &   &   \\ \hline
      8 &   &   &   &   &   &   &   &   &   &   &   &   & X & X &   &   &   \\ \hline
      9 &   &   &   &   &   &   &   &   &   &   &   &   &   & X & X &   &   \\ \hline
      10 &  &   &   &   &   &   &   &   &   &   &   &   &   &   & X & X &   \\ \hline
      11 &  &   &   &   &   &   &   &   &   &   &   &   &   &   &   &   & X \\ \hline
    \end{tabular}
  \end{small}
  \caption{Cronograma das atividades previstas}
  \label{t_cronograma}
\end{table}

